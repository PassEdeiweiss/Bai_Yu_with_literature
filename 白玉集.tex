\documentclass[a5paper]{ctexart}
\usepackage{graphicx, url, float}
\usepackage{geometry}
\usepackage{amssymb,amsmath}
\renewcommand{\d}{\mathop{}\!\mathrm{d}}
\newcommand{\e}{\mathrm{e}}
\renewcommand{\i}{\mathrm{i}}
\newcommand{\R}{\mathbb{R}}
\newcommand{\C}{\mathbb{C}}
\newcommand{\N}{\mathbb{N}}
\newcommand{\Z}{\mathbb{Z}}
\newcommand{\arsinh}{\operatorname{arsinh}}
\newcommand{\arcosh}{\operatorname{arcosh}}
\geometry{a5paper,left=2cm,right=2cm,top=2.5cm,bottom=2.5cm}

\title{白玉集}
\author{徐白玉}
\date{January 2024}

\begin{document}
	\maketitle
	\thispagestyle{empty}
	\newpage
	\setcounter{page}{1}
	\tableofcontents
	
	\newpage
	\setcounter{page}{1}
	\section{序}
	我本来不想这么早就把自己的作品整理出来的。在2021年的冬天,我已经写了很多篇作品的时候,我还觉得应当在自己大限将至的时候整理自己的作品。让我改变了这一想法的契机有两个。一个是在我2021年的生日,我在原班的一位朋友整理了我发到 QQ 空间的所有作品到一个笔记本上,作为我的生日礼物,扉页上还贴了一朵花;另一个则是在第二年的夏天,我的挚友给他身边的几位爱好创作的朋友各整理了一份作品集,当然有我的一份。这时我才发现,原来自己的创作已经这么多了。再这么写下去,可能在我大限将至的时候根本整理不完。于是我觉得有必要整理自己的作品了。这才有了这本《白玉集》。这本集子基本收录了我所有的古体诗词、文言散文和现代诗。对于我给高中班级写的班史部分和我为一些老师写的人物小传,由于涉及隐私,我只挑选了一些合适的段落放上来;对于我高中时没事瞎想写的动物传记,由于内容比较幼稚,我就不收录在这里了。
	
	我一直对人说,我高中的三年可能是我诗词创作的巅峰。这不是因为那时候的作品写得有多么好,而是因为我在高中第一次有了\textbf{格律}的概念,这要感谢吉林一中的黄河老师。回头看我高中之前的作品,我觉得都是简单的“连字成句”,尤其是那首《伪满江红·教师节赋》,现在我都读不下去,因为实在是太烂了。所以各位看官在斧正这本集子的时候,应该可以看到我在格律上从无到有的一些进步。也就是说,如果您在前期看到了一些屑作,请忍住不要摔书,我相信后边的一些作品还是会让您看得舒服的。值得一提的是,在我了解了格律之后,我一直都在“用今音填古词”,甚至在2024年,我甚至还用英文填了古词。我觉得这没什么大不了,毕竟苏东坡先生都能“嬉笑怒骂,皆成文章”,而且与时俱进是我们应当遵循的发展理念。
	
	我的写作有一个很大的缺点:我不是很会以景写情。所以看官们在斧正我的诗词的时候,应该会很少看到我使用别致的景象去抒发什么。我知道这是我的缺点,因此也写过纯景的骈文,如《奥森游记》,但是景和情的分离性还是很高的。所以别看我写了这么多东西,我本质上还只是一个入门级的学徒,还有很多东西要学。我的原则是:在写中学,边写边学。
	
	如果您是一位想要学习如何写作诗词的同好,我觉得您必须要知道什么是格律,而且尝试着遵循格律写一些作品。我推荐您阅读王力老师的《诗词格律》,并购买一本上海古籍出版社的《唐宋词格律》作为您的写作工具书。格律这个东西虽然不是诗歌创作中一定要遵守的,但是在我看来,不守格律应当是以守格律为基础的。从我的个人实践来看,一个作者,只有在格律诗写得顺手了之后,才能跳出格律的框架,写出优秀的不守格律的作品。
	
	如果您使用过人教版的初高中教材,您应该可以在我的这些作品中找到您学过的一些课文的影子。因为很多作品其实都是我在学了一些课文之后,仿写了它们的格式用以表达自己的观点的。比如,您可以在《过吾论》中看到《赤壁赋》的铺陈技巧,可以在《春草歌》中看到《琵琶行》的叙事模式,可以在《以我》中看到《劝学》的比兴手法。
	
	对于一些我觉得写得还不错的、或者是有深意的句子和词语,我都会使用黑体标注出来;对于一些值得说明的创作背景、一些不好懂的内容和一些化用其他作品的内容,我也会使用脚注写在当页的下方。欢迎看官审阅、品鉴。
	
	\begin{flushright}
		 徐白玉作于吉林家中
	\end{flushright}
	
	
	\newpage
	\section{2017}
	\subsection{讽刺诗四首}
	\subsubsection{一}
	\begin{center}
		正襟危坐桌椅间,\\
		笔墨纸砚手中现。\\
		何事要用此四宝?\\
		传小纸条不传钱。
		
	\end{center}
	\subsubsection{二}
	\begin{center}
		没有教师没有卷,\\
		没有考试数不完。\\
		三十年后人皆富,\\
		独有一人苦搬砖。
	\end{center}

	
	\subsubsection{三}
	\begin{center}
		过河拆桥走捷径,\\
		忘恩负义是本性。\\
		既有壮胆行劣事,\\
		何惧人称狗之名!
	\end{center}

	
	\subsubsection{四}
	\begin{center}
		温室花朵真美丽,\\
		不热不冷肥料吸。\\
		路人轻折根未断,\\
		养花之人心却急。
	\end{center}

	
	
	\subsection{雨晴记}
	二〇一七年八月十二日,自辰时予初醒,则直为阴矣。阴云密布,阳光不见。后巳时始雨,屋舍黯淡,光彩全无,地陆块湿,稠土粘脚,树木低垂,伛偻驼背,雨点透窗,阳台沾湿,弱光满室,昏暗如此。予一生独爱润地之小雨,故心不说,于桌前作务,至未时方休。
	
	时毕,予抬头,忽觉室内光亮,心奇之。透窗而望,见天上晴朗无云,万里清净,蔚蓝如海,澄澈似水。忽有鸟儿越片天空,巧画优弧,面东而走,翅翼展舒。轻视耀日金碧辉煌,无限烂漫,撒地荣辉,送地热光。环舍物者,屋舍见日,亦显金黄,地上水蒸,土燥善行,树木挺直,生机勃勃,予心舒也。作务之乏,乃皆散也。
	
	夫高日壮丽,将人舒畅,而夜幕不远,一片黑城,庸灯闪而不可如日,心不余乎?则为人者处世,不可碌碌,少可习知,名可不扬,青可有一技之长,中可为国效力。虽老而薨,亦名耀青史,千古流芳,君子也。其德嶷嶷,其动也时,世皆佩也。则士,必献大局,烝烝而治,则实可受服者也。
	
	以上所云,皆我足己耳!
	\begin{flushright}
		2017年8月12日
	\end{flushright}
	
	\subsection{伪满江红·教师节赋}
	不骄不躁,两肩宽,载上一切。意高远,初三既至,岂容松懈?循循诱导显师道,谆谆教诲尊己业。惊回眸,念三年时光,何以谢?
	
	少年愿,难泯灭;现代子,求飞越。忘却苦,但求青黄相接。夏日不顾阳炎热,冬季何惧风凛冽!待一年,锦上添花时,绽笑靥。
	\begin{flushright}
		2017年9月9日
	\end{flushright}
	\newpage
	\section{2018}
	\subsection{清平乐·一模前}
	闲杂时少,不闻白丁扰\footnote{刘禹锡《陋室铭》:“谈笑有鸿儒,往来无白丁。”}。铁栅篱封数尺草,春暮生机未了。
	
	月出灯火通明,悠悠照我前行。室外云归雨暗,心中雾散天晴。
	\begin{flushright}
		2018年4月17日
	\end{flushright}
	\begin{flushleft}
		\textbf{附赵俸翊和词《伪望江南·酬白玉一模前见赠》:}
	\end{flushleft}
	
		一模前,心惶仍需坚。瀚海扬沙困齐桓\footnote{《韩非子·说林上》:“管仲、隰朋从于桓公伐孤竹,春往冬返,迷惑失道。管仲曰:‘老马之智可用也。’乃放老马而随之。遂得道。”},宁远势孤再无援\footnote{《明史·袁崇焕传》载:崇焕守宁远,力据孤城,击退努尔哈赤。}。愿同徐梦圆。

	
	\subsection{清平乐·母亲节}
	正鼻青眼,黑发微圆脸。几载辛劳眉色浅,倦意凹凸难掩。
	
	一年一度佳节,阖家团圆相携。明年再逢此日,忆今中考夺捷!
	\begin{flushright}
		2018年5月13日
	\end{flushright}

	
	
	\subsection{毕业诗}
	\begin{center}
		别离六月闷,夜半难将息。\\
		瓣簇湿碧草,日升唤雄鸡。\\
		锡离小兵体\footnote{语出安徒生童话《坚定的锡兵》。},金褪王子衣\footnote{语出王尔德《快乐王子》。}。\\
		秋树弃夏叶,冬冰融春泥。\\
		三季从中过,两山今已移。\\
		人生自亮丽,无意顾东西。\\
		鸟雏初破卵,更盼母归期。\\
		鸟雏离巢去,却喜云脚低。 
	\end{center}
	\hfill 2018年5月14日
	
	\subsection{做数学难题有感}
	\begin{center}
		阴云密布灰枝深,薄雾纷纷昧理真。\\
		琐字繁繁废纸页,墨油点点漆吾身。\\
		林公身老蛮荒地,崇焕兵穷宁远门。\\
		莫将乏困抽神矢,学术须得满面春。
		
	\end{center}
	\hfill 2018年12月21日
	\newpage
	
	\section{2019}
	
	\subsection{闲诗}
	\begin{center}
		
		日晴天冷更多风,簌簌横吹吾老翁。\\
		眉皱难平珠似血,脑疼不褪发如蓬。\\
		霞客随心留笔记,青莲厌第恳真朋。\\
		我心自有豪情志,无那当今未可能!
	\end{center}
	\hfill 2019年1月3日
	
	\subsection{与刘佳和唱和而作}
	\begin{center}
		只要基础知识掌握牢\\
		每一道题都经得起推敲\\
		如果你要问我这题出得好不好\\
		我会说这题固然好\\
		只是做起来让人受不了\\
		面对这一切我们应该如何是好\\
		或许时间不是最好的解药\\
		所以请不要倚着墙角给老师做报告\\
		涕泗横流的说着我考试没抄\\
		如果身边的同学总是吵吵闹闹\\
		没有人知道你的烦恼\\
		那就饮一壶杜康把它忘掉\\
		一醉方休走上人间正道\\
		不被那世俗的羁绊缠绕\\
		让李白成为我们的骄傲\\
		让好好学习成为一种高调\\
		别让难题乱了你的阵脚\\
		也别让简单题成为你的雄关漫道\\
		把一分不浪费作为目标\\
		谁说努力不一定能让人提高\\
		我就在那暗淡的星空中明亮地闪耀
	\end{center}
	\hfill 2019年1月7日
	
	\subsection{半梦中作}
	\begin{small}
		序:余适新春,欲有诗以庆之,然文辞阙漏。及日前作五言诗,久未有能言之辞。昨夜梦中,见余所慕者赠诗与余,中有佳句。余不能遍诵其诗,故以能诵之佳句及余所添拙句,于是补全其诗。
	\end{small}
	\begin{center}
		新节既至四传锣,问我年辞何未脱。\\
		不是作诗词必少,但因下笔话难多。\\
		盲书文事惟闲业,细阅佳书方正活。\\
		壮士自存斩浪志,黄金榜上闹风波。
	\end{center}
	\hfill 2019年2月8日
	
	\subsection{过吾论(其一)}
	雄起向年,三月之暮矣。乍暖之际,如常一试,以测也。及毕之,予之境不可善观矣。故书此文,以言其咎。
	
	试者,其题目难易繁杂,不可细言。然多长于吾者。吾谓之曰或直佩,或不直也。直佩者,能至余之未竟也;不直者,况师有授焉,孰不能为?今徐子之不能为者,时误耳。
	
	由是徐子有叹焉:“师每曰:‘所通者不能得其分,似未通也。'其有理乎?纵其毕之同,然相距远矣。未通者,无知无德,年与时驰,遂成枯落,多不接世,终可悲可叹者也。通而不得者,有知有德,愿倾毕生之力,壮理想之业,何犹之有!
	
	“夫人之成败,岂可朝见而暮曰知之乎?复能所以一文一献而云知之耶?由是世间众人,成者不一定成,败者不一定败,此由己也。人之奋进,非纸笔所能尽显也,何言成败哉?今之二三子,见人之不明辄曰:‘败矣。’此非愚,乃大惑哉!”
	
	呜呼!知此理者,世间几人哉?晓此道者,四海之大有几人欤?唯吾与公\footnote{公者,读此文者也。}能有所共勉也。
	\begin{flushright}
			2019年3月25日
	\end{flushright}
	
	\subsection{古体(其一)}
	\begin{center}
		世中有异士,自名第一狂。\\
		青莲欲与比,自愧不若强。\\
		朝驰在田野,笑露终为霜。\\
		暮见天色晚,呓语鸣山冈。\\
		梦中忽得翅,振翼在八荒。\\
		羽扫千万户,洁净复清凉。\\
		次朝归故里,骏马带斜阳。\\
		心愿能再旅,日日西向望。\\
		待到马肥时,再出骋袤场。\\
		命中欲足己,不必有金觞。
	\end{center}
	\hfill 2019年4月3日
	
	\subsection{送孙梦聪序}
	予自从师而学,每思之曰:世中男女之别,可谓深矣。古人之鄙于女者,是致于男能尽于阳刚,女能尽于阴柔;今人之平于男女者,是平于阳刚气焉。是故男有柔寡者,女有刚强者。女之刚强者,是孙君者也。
	
	予于初中,见恶于劣者,其予大刀阔斧之故乎!而今如一中,孙君似予初中一人,富态而易躁。其人之道,恶我之甚,非能片语之所述也。然孙君之心,待人以义,贤于其人远矣。是予之善与孙君故也。是故今有惜情能作此文也。
	
	夫孙君,其善也达,其义也充。初见辄坐于予傍。先常存怒气,久不致喜。而今之欲学也。稍品其性,多人之不能及其义也。时至今日,然终致于文者,须前时之谬乎!然是事人须叹惋耶?予以为弗焉。夫从文,新始也。既已有其心,何愁无成?
	
	嗟夫!今之一别,虽能逐路而见,亦如永别。愿孙君能保今之品性,奋发而上,由是则予今日之情不空泄也。
	\begin{flushright}
		2019年4月7日
	\end{flushright}
	
	
	\subsection{过吾论(其二)}
	白玉愀然,翊公问其故。
	
	白玉喟然曰:“我之过也,遍乎四方。实则非有超世之才也,而负戴甸甸之籍,局促狭狭之室,立乎济世之檐,并于知明之列,人悉能为我之不能为也。其知故耶?其奋之不及人故耶?我亦欲奋,不能进反而退矣。欲加奋,目眩而头昏,心悸而手废。由是我之位也,弗如让于人乎!”
	
	翊公曰:“谬矣白玉!是安合于过?夫欲成事,何须才也?何以知以才而济世,以性而知明焉?吾闻之曰:‘人生不如意十之八九,’何谓也?张子解之曰:‘励耳。不问八九,但求一二则已。’是有言于公之事乎?知之不及人,奋之亦不能为,公何以狂肆放荡而有今日之成哉?宽胖而不易还于柳腰,无言于不能减重也;目光逐窄而不易还于宽,无言于不能明目也;故退而不易还于进,无言于不能复上也。目眩头昏则眠,心悸手废则息。少之息而沉渊不返者,未之有也。如是苦位,何足让于人哉?公尝曰:‘人之奋进,非纸笔之所能尽显也。’其此之谓乎?”
	
	白玉大喜,并归于楼。
	\begin{flushright}
		2019年4月29日 
	\end{flushright}
	
	\subsection[赠五组]{赠五组\footnote{白玉时为14班第五组组长,时第五组将散,白玉作之。}}
	\begin{center}
		莫道今时人早谙,早谙身可在其端?\\
		残春难止棉虫落,初夏何防白日斑!\\
		晴阴交界朝虚度,风雨叠加暮未安。\\
		欲遣离愁往那处?长空雁叫娄山关\footnote{毛主席《忆秦娥·娄山关》:“西风烈,长空雁叫霜晨月。霜晨月,马蹄声碎,喇叭声咽。\ \ \ \ 雄关漫道真如铁,而今迈步从头越。从头越,苍山如海,残阳如血。”}。
	\end{center}
	\hfill 2019年5月4日
	
	\subsection{鹊桥仙·不想写字帖}
	茕茕蚯蚓,集集巢户,胆小岂能常驻?目极瀚海望无涯,喟然叹、茫茫前路。
	
	怦然心悸,猝然梦醒,但看夜光盈入。披星戴月点灯笼,借请问,人间何处?
	\begin{flushright}
		2019年5月10日 
	\end{flushright}
	
	\subsection{云阴日会父亲节记}
	\begin{small}
		序:维余之适高校,初夏,阴盈六日,间有雨而少露,有云而少阳。母见辞六日已。适父节将至,余就其事,感慨不已。特以文抒怀。
	\end{small}
	
	北部之境,东方之突。三省之中,四区之部。展宏图之有力地,尽鄙名之阙史处。车船新厂,顺雄鸡气噎之喉;宋金旧地,携二帝远走之陆。
	
	又日逢节,复年遇故。听笑语欢喜之家,闻无言寂寥之户。屋临二校,放眼皆孟轲之才;宇近三潭,极目得阮籍之路。既褪油砖,已露黑土。湿湿阻人之道,泞泞迈步之蹼。袭袭而至,微微轻抚之风;踽踽而行,颤颤难飞之鸔。
	
	绿变紫,白成灰。茕茕飞跃之兔,怖怖紧缩之龟。青草遍地,如怀善之使臣;黑云满天,似扑城之魔鬼。遥塔高立,近叶低垂。扬扬锣声,谁奏起兴之乐?哀哀长叹,孰存凄清之悲?岂知人与事,莫论是并非。何处寻轼辙日月?那里是俞钟山水?遥问吾客,能与吾诉赤壁之箫哉?
	
	余父杰,心宽体胖,身健气宏。性雅长志似林樟松柏,质洁本心如芰荷芙蓉。常语与鸦,通其母意;久坐同雀,晓其父躬。一句恶言,喜怒不效眉上;半边风雨,冷暖尽收腹中。榻边受业,愿承育儿之辛;沙上举鼎,能笑行役之苦。厨炙之力,心口从而意欲劳;工电之行,形体顺而情愿辅。是故今时不同往日,余之情乃不可不泄也。
	
	余白玉,成形于腹,赐名于天。无所不同之于人也。盖有人之喜悲,物之生死,非长时而欢乐,正短间之期许。故今日借笔指间,墨之能为泗隐;请纸桌上,手之可代头昏也。
	
	呜呼!墨笔终下,珠泪难已!举家之情,岂一言而抒悲?蜉蝣之节,何欢笑而露喜?况善意直情,优功厚济,本造化之赐物,何面目而归己?俊杰韬韬,英才难比,吾家之遽所求也。余比得接其心,承其意,立龙门之闩闭,待涨水之钥启。由是则真命可达,夙愿可成矣!
	
	嗟夫!心言巨,纸面小,来日方长。余有务于肩上哉!请加奋,以遂之云尔。
	\begin{flushright}
		2019年6月15日
	\end{flushright}
	
	\subsection{有感}
	\begin{center}
		心脆腰身壮,志坚金银失。\\
		妍花土中润,眠蚁洞里湿。\\
		荣草弃蚯蚓,成雏舍木枝。\\
		绪杂未可断,妇巧亦难织。
	\end{center}
	\begin{flushright}
			2019年7月2日
	\end{flushright}
	
	\subsection{做化学难题有感}
	\begin{center}
		莫言学术不须难,寒岁中间遇韧兰。\\
		力破一题师草圣\footnote{即张旭。杜甫《饮中八仙歌》:“张旭三杯草圣传,脱帽露顶王公前,挥毫落纸如云烟。”},猝得双解无潘安\footnote{潘安,又名潘岳,字安仁,荥阳人;西晋文学家。传说长相极其帅气。}。\\
		退之伏枥恨奴侮\footnote{韩愈《马说》:“故虽有名马,祗辱于奴隶人之手,骈死于槽枥之间,不以千里称也。”},介甫困奸登峻山\footnote{王安石《登飞来峰》:“不畏浮云遮望眼,自缘身在最高层。”}。\\
		壮士豪情慕悫浪\footnote{《宋书·宗悫传》:“悫年少时,炳问其志。悫曰:‘愿乘长风破万里浪。’”},便同高祖出乡关\footnote{刘邦《大风歌》:“大风起兮云飞扬,威加海内兮归故乡,安得猛士兮守四方。”}。
		
	\end{center}
	\hfill 2019年7月13日
	
	\subsection[古体(其二)]{古体(其二)\footnote{融合14班六位老师的名字。先赵俸翊有诗曰:“骄霞衬莲东,菡洁晓微风。碧荷成双送,潜影星宇中。”盖融其班六位老师之名。白玉以此和之。
	}}
	\begin{center}
		老马固识路,日夜望海滨。\\
		其希未能全,怅然耳目新。\\
		过磬猛一击,伏枥如无音。\\
		走辙微轻去,昂首阔步临。\\
		行人惧其异,就近问于邻。\\
		“隔壁何作怪?何为见吾亲?”\\
		“行人既已过,自有顺程心。”\\
		行人大慨叹,良骥有酸辛!
	\end{center}
	\begin{flushright}
		2019年7月15日
	\end{flushright}
	
	
	\subsection{清平乐·梦中忽起学意}
	\begin{small}
		序:余昨夜入梦,却遇一中试。会余病,固辞。为课之日,李顺成公曰:“吾班自有高分者,生物之能年级数一。”并直呼其名。余闻之,阅其卷,惊其题易,遂有所不悦。心火燎燎,于是醒。有感作此词。
	\end{small}
	
	青门皆掩,师言醒吾眼。盛赞子龙闯伏点,汉升决然寻险\footnote{事出《三国演义》。}。
	
	近临开学佳节,徘徊愤慨弗绝。配意动吾心悸,欲擒猛兽孑孓。
	
	\hfill 2019年7月23日
	
	\subsection[学中奇人歌]{学中奇人歌\footnote{仿杜甫《饮中八仙歌》为14班学霸赋焉。}}
	张宇交道题目圈,颇受盛名性仍谦。
	
	林君闻课似从仙,白丁纷扰皆无言,如常一试惊四筵。
	
	鹏飞几欲冲前沿,思维深广如无边,不惧难题并苍天。
	
	姝力为数甚清闲,未尝拒人桌角前,识者尽称世中贤。
	
	星燃神秘自家园,猛然好绩不四传。
	
	徐子遇理口流涎,叩首称兄马公前。讲台之上作疯癫,洋洋洒洒千万言。
	
	骚王做题不欲眠,长叹时间不如钱。
	
	于丹豪放马上旋,学中寻得再少年,劲头一横惟留烟。
	
	张姐能读万卷篇,宁静淡泊守己恬,俄而抢得他人先。
	
	王楠勤学懂钻研,为数稳如静水船,偶然一落无须烦。
	
	羿佟勇能登泰山,常如鲸鱼水下潜。
	\begin{flushright}
		2019年8月5日
	\end{flushright}
	
	\subsection{沁园春·贺楠君母难日}
	数日风沙,鸟雀去巢,游子离家。叹浩然书山,难得消遣;无边题海,久占闲暇。艳名之求,金邸\footnote{艳名、金邸,谐音双关。盖校长并年级主任也。}所愿,惟教莘莘加匮乏。苦秋近,问愚公精卫\footnote{映前文“书山题海”。},几度年华?
	
	楠君知慧足嘉,母难日恭福须不差。愿为学途路,畅通行驿;踌躇阡陌,清闲无压。喜乐长随,伤悲不见,步履悠悠到梦涯。诚如是,虽三头六臂,余何足夸?
	\begin{flushright}
		2019年9月4日
	\end{flushright}
	
	\subsection[春草歌]{春草歌\footnote{春草者,以喻成绩也。}  \footnote{白玉时初入25班。《二十五班班史》记:“满腹离愁,乃为诗《春草歌》、文《论适》以求安”。}}
	暖城毗荡秀毛犬,昂首前行悲不显。无谓食于乞者后,未求饮在友朋先。肌腱常奔强有力,挚同久伴密无间。自有傲心胸骨里,骄然睥睨刀枪嫌。
	
	幼子戏于暖城外,惜其秀毛恐相害。抬臂顿足惟欲吓,莫料闲庭信步来。异色面言长者与,尽称犬中有奇才。循子寻得奇才处,静立闭齿头仍抬。远抛肴骨足固停,近挑下颚容不改。人众抱起如捧月,长叹久屈街尾埃。猝得人间巨错爱,小犬惊惶颜焦哀。
	
	皂浴汤淋白如墙,坊中八面扑鼻香。生来初得如是爽,是夜眠得一觉康。喜意满面头上冲,提心今日胆下放。始教家中易作乐,久令盗贼难破防。互生期年欢喜至,可知今后何为伤?
	
	牵链日出路上行,路人皆以秀毛妒。家中所宠地下土,路边行者天上物。欲以千万换其物,惟言不可无予故。
	
	系链柱上人小去,不知觊觎来路至。损羁断链捧而归,归时云遮半边日。铺蔷盖笼定家户,撒粮设帐供食宿。少食多眠瘦如柴,望断路边松柏树。
	
	小犬窃自久欲逃,默思归家路未迢。笼开乘势出奔走,未尝得行一里遥。前后二院百余步,肠断人追捧起处。归途人悉喜无度,惟有小犬心寒酷。
	
	俦犬见其日闷闷,惑心颇起见与论:“尔本路边浪荡犬,幸入暖城存此身。存身自有惬美地,何故单恋小宅门?”
	
	小犬面色遽以灰,起言语调不胜悲:“君是城内本家犬,岂知孤士理无亏!君不见笼中丽鸟恋故林,瓶里小虫思翠微;君不见过客去家泪不止,游子离乡鬓直衰。年年春草绿如油,日照中天起光辉。正值西风清冷季,小犬明日能西归?”
	
	话毕俦犬接而道:“萋萋绿草阻人归。”
	\begin{flushright}
		2019年9月10日
	\end{flushright}
	
	\subsection{论适}
	人行其事,必求其适。事适人意,可以常乐,心既正,意既诚,弗苦事之不平也;不适,倦倦于世,无以进取,隐形蝼蚁之群,局促蜉蝣之间,何以志?
	
	人生于天地,所遇人物多矣。欲得尽适,诚为难矣!然则欢朗之人,心宽如海,胸阔如原,日欣月喜,如无所忧者,何哉?命中无舛,万事顺意者,未之有也;通达学识,晓所应为,以求适己且物者,盖此人之俦也。
	
	呜呼!理何其易,知者亦多,用者甚鲜,何故?喧喧之世,人不能谨守也;碎碎之知,人不能寻路也;懒懒之行,人不能从继也。其人之所以不适乎!且夫世物之杂,人何所适?何以知之?是故所以人之不行,怨世弃道,浑浑终日者可知矣。
	
	嗟夫!余为学所见不能适新者多矣,思夫余亦尝有不适之行,方有所思,足己而记此文,盖勉世中之不适者并适而少茫然者所云。
	\begin{flushright}
		2019年10月20日
	\end{flushright}
	
	\subsection{做椭圆大题有感}
	\begin{center}
		扁平殊物似长布,一日欢欣尽抹除。\\
		正解不得牙溢血,顺思难觅发将秃。\\
		但学至圣演穷厄,莫效青白哭困途。\\
		休待同窗人四去,万行涕泪归茅庐!
	\end{center}
	\begin{flushright}
			2019年10月21日
	\end{flushright}
	
	
	\subsection{水调歌头·侃制}
	赵将暴食日\footnote{《史记·廉颇蔺相如列传》:“赵使者既见廉颇,廉颇为之一饭斗米,肉十斤,被甲上马,以示尚可用。赵使还报王曰:‘廉将军虽老,尚善饭;然与臣坐,顷之三遗矢矣!’赵王以为老,遂不召。”},德祖命亡时\footnote{语出《三国演义》杨修之死。},扬雄笔下千语,仲永幼能诗。李广腾飞之绩,杨氏倾国之貌,湮没谁人识?存手尚偏爱,尽弃亦可支。
	
	秦始皇,明洪武,四方驰。毕生功业,消泯孙辈几人知?任尔金戈铁甲,更况刀枪剑戟,结落数城尸。休道囫囵闹,我岂乱言痴!
	\begin{flushright}
		2019年11月28日
	\end{flushright}
	
	\subsection{以我}
	冰雪,以其寒也,乞人憎之;亡江之春,化为水,豪户欣焉。文事书具,于不识丁者弗用,而于士则可以取功名也。沙场奔兵,人皆笑之,然人岂皆不奔乎?血难疏离,妇泣妪啼,其人亦悉知也。是故人言“旁观者清”者,许不敢贸定也。猫在路,见鼠而捕,然弄鼠隐其后而随之,随止随行,猫不顾则无以见矣。故曰:盲猫无以捕食,浅人无以知世,其在己也。
	
	由是则人之于物参差者可知矣。伯牙之幸甚,遇子期之耳,然则万里志士,欲寻其共者其难也!夫千人观一,共者几何?豪户车游,横罹难死,或以人命而戚,或以豪灭而喜,人事迥然,其如常矣\footnote{“豪户”一事,盖语文学科《小题精练》中所述。}。国,人治也。人之思异,则各国异,然各不能破者,人适国而国亦适人也。是故西学之难东盛,外法之难国用者,可知矣。
	
	或曰:“燕雀不知鸿鹄,鸿鹄不晓鲲鹏,鲲鹏不羡天地,天地不容宇宙。”然则小大之悲,实则有之哉?燕雀游春,混同天地灵性,惟自适矣;无所悲,亦无所羡者,其所有已有焉。夫神龟行命,睥睨走兔,移步为景者,其悠然自得也;然兔之知乎?龟求其知乎?亦自适而已矣。今有扬言各物异同,伪较以谏而实则大挫人心者,不亦颠乎?
	
	行者于路,以步为马,驰千里之波不顾;以目为笔,绘四海之图不倦;以念为刀,斩一路之棘不止。虽见笑于人,以为不共人者,复能几何?王阳明云“此心光明,亦复何言”者,其此之谓乎?
	
	呜呼!闻谲言,能不空叹谬比,而抨之以理者,余所共者也。然其谁欤?其弄鼠欤?其燕雀欤?其神龟欤?噫!其有人乎?余但求一人可矣!
	\begin{flushright}
		2019年12月4日
	\end{flushright}
	\begin{flushleft}
		\textbf{后挚友赵俸翊以此题立意,为诗歌《以我》。附之于下:}
	\end{flushleft}
	\begin{center}
		书读十载常觉庸,笃志半似无用功。\\
		本是寄怀年少事,东隅已逝怎追空?\\
		生来自适执孤远,泉涌情思口难开。\\
		静郁不合嚣放客,乱花渐欲芷兰栽。\\
		斜阳朗照彩云沐,一念回眸百念脱。\\
		能侃能诗天地阔,且行且歌方正活。\\
		烟波去去人能何!言自心刻久不折!\\
		谁笑多情稚子梦,流光璀璨荡星河。\\
		我从去月化学定\footnote{指赵俸翊化学竞赛获奖。},屈指龙门二百遥。\\
		遨想群书应岂料,不枉来此走一遭。\\
		落花风雪年年归,题典无穷头不回。\\
		莫恨轻颓空怨怼,万难既克谁人摧?\\
		送还明月来迎夜,一笔一人一世界。\\
		甘作拂衣孤胆杰,管他江湖我孑倔。\\
		寒山雪满几捱冬,往来壮士皆英雄。\\
		时势弄人末路勇,敢问成败是志穷?\\
		学非通法有通途,天不绝人无绝路。\\
		意气正当挥遒时,俊青才女吾辈出。\\
		遥怜沧海多明珠,别叹光彩君不如。\\
		薄暮无言霜满目,此生望断天涯路。
		
	\end{center}
	
	\newpage
	\section{2020}
	
	\subsection{简笔贺翊公生日}
	\begin{flushleft}
		翊公谨启:
	\end{flushleft}
	
	余以无油之肠,胜火之心,欲贺公之生日。袋中之物\footnote{一个滕王阁的模型。},盖余同父母同心选者,而许不合公意。是切见谅。
	
	思夫吾二人互知者四年有半矣,其间多事。彰情者千手难指,而劾恶者屈指而清。许古汗青之亿万万字难觅矣。复有去年此时,余意杂乱,学务冗繁,故车上为乱词\footnote{一首未按格律写的《西江月》,遗失不录。},其词笔轻,其言寡淡,其形糙旧,其衰也。而及余之母难日,公以美文贻余,情理通达,行文流水,此余十年之难及矣。见字如面,如面则蔼,知蔼而愧。所愧者,余之不能和以共文焉。故今日为之,笔墨滞塞,行楷简陋,诚愿公闭目视之,由是余之可勉矣。
	
	一年间,我等得其勉度,循意而行,故互有其成。余之得,公之明鉴,余大可不必言之。而公文理所长,渐学渐进,新知成赋,老道恒忆,终摒初如学之寒气,日攀于峰,芝麻如也。以是今日之公,将循便日之旨,踏牙突之步,闲如高府而学\footnote{时翊公将与郝一鸿等进京学习化学竞赛。}。是视如桥景,风光无限,而水内之波涌,几人知之?公以此行,贻误生日,此命中如阙一期矣。余亦不可择其日礼公,故择当日为之,以贺其日之近,并践此行。幸甚一鸿之随,余既属之侍日。公旦落县心,毋以为远方亡家之夜亦失友以随,并知或能知公之日者。
	
	呜呼!寥寥几笔言,盈盈无限情。其情此意,岂笔之能书?而公早合余之心,此意字间则知。一年间,高三至。公莫复知始,惟怀抱功绩,奔勇千山,遨海逐洋可矣!
	
	笔如是尽。以是余能以和安之心,迎余之日焉。
	\begin{flushright}
		2020年1月13日
	\end{flushright}
	
	\subsection[与余昕泽对成]{与余昕泽对成\footnote{白玉与余昕泽对诗。一联之中,昕泽为上句,白玉行下句。得本诗。}}
	\begin{center}
		林深幽抑夜归路,青墨长天冷煞人。\\
		亭宇伫人独煮酒,路途去客暗伤神。\\
		闻笛却步无言意,抬袖擦肩有宿尘。\\
		风过拂纱是老母,伏身香梦捣衣砧。
		
	\end{center}
	\begin{flushright}
		2022年2月18日
	\end{flushright}
	
	\subsection{疫间奉师命行日记录}
	\subsubsection{一}
	朝,起食。食毕待试。适余母在家,督余为之。
	
	午,余母为甘旨。食毕,憩少顷。午时末起,打卡。
	
	午后,阅英语以备试,颇得新道。及试,余笔书从速,手指始痛。时凉风入室,余头晕不止。
	
	故试毕稍息,持至宵食。
	
	夜,听网课。
	
	今时,余仍头疼不止。余阅新冠之症无之,暗自庆幸。许唯一眠可解之也。
	\begin{flushright}
		2020年5月14日
	\end{flushright}
	
	\subsubsection{二}
	朝,起而为食。思夫昨夜之未尽,意不平,虑久,不得其略。及试,以诚信之言,虽父母之忙务,余何敢恶念!试毕,跳蹴屋中,故曲为伴。久之,汗之不出,以为奇。毕,读书。
	
	午,肴尽早。初欲小憩。而乍现不动之繁务,昨夜之不解,何以稳眠哉?于是四求四问,推敲不止。忙及申时初,化学起,适余务尽。化学尽,观夫今试间,物理不拔,何及悲叹!刷题而已。
	
	晚肴尽,辄见金公。晚课断续。其间兼动兼学,至于所动,唯跳走耳。
	
	时将与余之俦者论数,平生也,俦者不数。然乐而好学者鲜,余之幸甚,幸甚!
	
	余闻:“不动难眠。”此言善。自疫情起,余常能覆口鼻,外行千余步,当夜之梦皆不胜善也。今家乡之急,余平日不行,局促一室之内,定身半檐之间,所能为者,岂唯跳走乎?非也。是故明日有所求矣。
	\begin{flushright}
		2020年5月16日
	\end{flushright}
	
	\subsubsection{三}
	朝,卯时中,起。
	
	阖午前闻课,欣然无空闲。人知余怜诗词之旷丽者,许能晓今日语文课上余之雀跃也。杜工部之《秋兴》,愚以为纵一首之难解,不掩其三四首之美不胜叹焉。若余皆有语文之乐,许不能退也。初,师叹:“生物原题,何以再误?”中有余一。思之,余非皓首不学,难治愚顽之士,且夫少幼之时,一而不再,再而不三,何至今日?其中倒退,余愿求之反之,抨之记之,但求不三。
	
	午后无不常,新知既受,不就多言。而别离数月,思忆极甚。前日之梦,与友共勉而学,为杂杂题目,久而不倦者,心之相也。故今日能见,许余心见知乎?荣幸已至,余无可求。至于晚试,不必多言。
	
	先是,公常言我等之不动。今日余极为之。跳蹴之属,亦必有之;撑身狭局之室,抬腿阔敞之厅,其时多矣。则明日何如?尚不知,且观其然云尔。
	\begin{flushright}
		2020年5月18日
	\end{flushright}
	
	\subsubsection{四}
	朝,起早。余闻:“卯时醒,不可再眠。若,必倦于阖日。”故不敢眠。
	
	午前,欣然为学。课间为化学。闻夫生物所授,余竟不能忆也。慨之。
	
	午,食毕,不欲眠。为学多时。
	
	午后,献读《将进酒》。得邓公之盛赞。一日之乏,此时辄尽。
	
	夜试。选填得朱衣点头。而二大题者,皆不得也。何其病焉!
	
	手书结语之时,适余之务早尽。尝以师督,不敢妄为。故读书。今日之读书,余方受益匪浅。先是,余不爱人,至于不属,多有鄙夷。今之其谬矣。昔仲尼之辩“华夷”者,自恋而已。自违其言:“阙如于所不知。”余之行岂不类之乎?盖好书之言,皆珠玑也。
	\begin{flushright}
		2020年5月19日
	\end{flushright}
	
	\subsubsection{五}
	朝起。自集表\footnote{时疫间,每朝七点,班主任焦子命同学签到,以为起床之行,是为“集表”。有签毕辄眠者。}以来,余皆能骄然命笔曰:“必诚。”是故有所限。人言:“心中有限,必较人生之得失。”其此之谓乎?
	
	午前闻课,不别于往日。
	
	午,目乏欲眠而心不欲。适午后师许余歌,兴奋之至。劳然之意,泯不复存。快欣之感,持之是时。夜试。自余如高府,不殆英语,而就不能大成。今有其得,愚以为吾师之正道也。得师如是,幸甚!
	
	余今日以读书不快。作务早尽,本欲读书久之,而心不达焉。是故命笔而作,欲效古人之诗,而落笔知难。思夫余如校之日,日日早结而为所欣。诗词歌赋,信手拈来可矣。今日何如?余之退矣,明明乎于所欣,教教然于所爱,何其愧也!
	\begin{flushright}
		2020年5月20日
	\end{flushright}
	
	\subsubsection{六}
	此日善极。朝,破乏而起。午前闻课,并无所别。
	
	午,余父母为火锅。思夫余不尝一味吾之校也。午后试生物,余为题其速,而能定其确。此余之初也。今务何其轻也!是故为之毕早。小误于音之不能止,人皆听之。若非师言,阖事皆见知于人。不知可谢。
	
	夜试,余求其缓,得而不达。
	
	近与俦者论议为题。余以为倨。尝币九重之书,而过西风之境;乃散烟火之银,兼破荆棘之城。手把芙蓉之论,怀抱黄金之书;口吐磅礴之语,笔走深渊之龙。余爱数理化,风流人不知。欣然进我境,何顾求闲时?是故愚以为可小满焉。
	\begin{flushright}
		2020年5月21日
	\end{flushright}
	
	\subsubsection{七}
	却曰人之喜悲,必有所定焉。先日,余面覆喜笑之色,步踏红木之地,其情欣,其意健。今朝得不快之矛,夕举不得之盾,维今时如夜,能有所安。
	
	前日欣至人得,其语即见,致不记交与所为。朝师督之,余方愧然。其欢喜而害人乎?其自满而损人乎?小事见之,不须犹疑。然则乐极生悲之事,世中之必然乎?非如是。是故人之乐也,各守其度。
	
	午后,目眩,政课尤甚,课毕稍减。复带不忍之腹痛,少顷而三矢。幸甚务少,否者必误事也。
	
	夜试,次余父。余父曰:“匪观尔之二三子,观尔也。”余为之,初有所困,至中,先难方不难也。盖余初读而不见其要。
	
	今日郁闷,余知俦者皆如是也。何以遣之?于余方歌。余初歌于组内,引磅礴之浪,动俦属之情。此余之幸乎?盖人间之事,不如意者十之八九。为之奈何?不问八九,但求一二则已。此余之所得也。
	\begin{flushright}
		2020年5月22日
	\end{flushright}
	
	\subsubsection{八}
	夜,蚊虫嘈杂,不能善眠。朝,辰时初,起。
	
	余爱读书,尤知者之书也。今日多读书。读书之余,为务不息。夜至今时,觉今日之行,知识多矣。
	
	生物试,试毕,余同二三子谈,闻余之善,喜甚。
	\begin{flushright}
		2020年5月23日
	\end{flushright}
	
	\subsubsection{九}
	朝,起早。待早课。此长假之初也。间动作,观夫所录,终有所忆。午后,英语代课。难答其问,愧。后,答金公之问,喜极。
	
	夜试,余以此题善,提笔大就,挥手而成。
	
	余近颇明诚信之义焉。先是,余常不会而寻之,而终不得也。今心平气和,别无所顾,方大有可为。盖余剑不磨,无以锋芒;余心不正,无以大成。网课之日,余竟各有所得,其余之多虑乎?其余之动天乎?
	\begin{flushright}
		2020年5月25日
	\end{flushright}
	
	\subsubsection{十}
	朝起早。思夫《弟子规》云:“朝起早,夜眠迟。”余不喜此书,而愿为其句者,欣然而已。
	
	或问余曰:“何以欣然阖日?”对曰:“不顾心中之魔,不观削直之剑,可矣。”又问曰:“何以佳绩?”对曰:“以诚为事可矣。昔余搜题而低下,而今信然为之,大有所成。故曰:搜题者寻其落矣,非后,眼前而已。”
	
	余知金公劳务,而语气欣笑,此怪也。余尝心不静,奏乐而起兴;而金公之日日早起晚眠,未见其泄,而笑意阑珊不灭也,此余九思之不解。
	
	夜试,正解难得,虽逐一试之,得。
	
	余今日读毕《圆圈正义》,以其可再读也。其言钉钉,其语切切,其理达达,其情融融。可九读而诵焉。思夫余读书以来,未尝抒如是之语,此余之幸乎?
	\begin{flushright}
		2020年5月26日
	\end{flushright}
	
	\subsubsection{十一}
	朝起早,备今日之行。设蚊帐。
	
	今日之对也优矣。师有所问,必能善对。此余以为昨日之更成。盖前日难得位置,畏惧而已。
	
	夜试,题难。余骤然不敢落笔。出,见所深顾者鲜其误,而匆疾者废矣。若能握松柏之本,摘香臭之果,则万试无须忧矣。是言并数。
	
	余有庞然书籍在校舍。今离之万里,求之不得。唯抱局促之书,以为三冬之火,终究得之者,盖开石而已。昔余父久而不识“研”字,余祖父曰:“开石为研。石不开,事不就。”其此之谓乎?
	\begin{flushright}
		2020年5月27日
	\end{flushright}
	
	\subsubsection{十二}
	朝起早,待早课。阖日,师无所问。思夫求师有所问者,许惟余乎?
	
	夜试,余先为大题,遂有所节。其题初见繁杂,而细究之,其易甚。
	
	余今心平和无动,无所大感,亦无所伤。附有倦意连连,故少写。人言:“少言而大知。”余愿能为之焉。
	\begin{flushright}
		2020年5月28日
	\end{flushright}
	
	\subsubsection{十三}
	余以亵玩之心、闲散之目、摇荡之手、难缄之口,为学久之。此所以余之有兴焉,而颇不疲累。夫余之见一中师也甚,皆未尝言于端正,中见多识广,勃兴与生,欣然为题,欢喜在面者,其生爱之而有所成。此孔子之谓“乐之”乎?
	
	今日之务也繁,繁而不难者,其道也。
	
	邓公问余曰:“‘六部’何?”余骤然对之。其事实有可言。余自幼喜绘,有人物图形跃然,是故必咸赋其名焉。由是以位起名,方得以皆诵。
	
	夜试,题易。
	
	余固爱史。是故语文之能有小成。余尝曰:“若余皆有语文之乐,许不能退也。”而今能达之,欣甚!辄毋销其致,学于陡然,由是必大有可为也!
	\begin{flushright}
		2020年5月29日
	\end{flushright}
	
	\subsubsection{十四}
	五月将逾矣。余方叹于时间之速。
	
	昨夜并父母有所言,至凌晨。故今稍晚起。
	
	为数,题不易。自幼余长者曰:“尔能通文意,许无所烦于此。”然也。而虽通文意而不得法者,其题过难乎?其余之恶乎?
	
	夜试,题有趣。余慢条为之,以为有所得。
	\begin{flushright}
		2020年5月30日
	\end{flushright}
	
	\subsubsection{十五}
	朝起早。是日也,余浴文海。初,余为文不得,条框之务,事事为限。自如高中,渐能信手理论,是余进也。
	
	程公赞余曰:“闻尔语,如余之师。”余惊而愧然。余幼时善绘,无理亲友之言,专注手法之强;少长,慕捕者之姿,欲治遍世之安,救水火之人,习胜战之术,除黑恶之贼;长,其事殆,欣于仁义之道,感于妙手之春,欲为医;时至今日,余观网路之言,鲜赞其师,或言“师有其指,无爱人者”。思夫余毕生所遇者,皆爱吾有加,于此言语,恸之恶之。故立志友其生,善其道,著己身之言,正本心之行,如此而已。
	
	今有雨而暮晴。余思之:夜幕之不远,其可用哉?故暮年之学,立于少时不学之上,悲也。余必加鞭而行,方有心梦之成。
	\begin{flushright}
		2020年6月1日
	\end{flushright}
	
	\subsubsection{十六}
	朝起早。课上,能答英语,较于昨日有所进。
	
	夜试,题易。余答之善,且实之焉。
	
	近暮,阴云密布,阳光不见。少顷,大雨倾盆,遍天土黄。时余桌前作务。先是一日,朝阴翳,午降雨,余亦作务,而其暮天晴。今之暮也,云覆碧空,日在正头不见,及落有光也。是日务少,故能凝目窗外之观。
	
	近于俦者论古诗文。每言之,余能开大口,流长涎,侃侃而谈,滔滔不绝。是余极欣之也。复思罗翔《圆圈正义》曰:“知贫者,其目何其小也,胜达不见,故自骄;知富者,其目远,知知之不能及也,无以骄。”是故余常言曰:“余不过爱之耳。欲求其技,余惟能期公亦能爱之也。”
	\begin{flushright}
		2020年6月3日
	\end{flushright}
	
	\subsubsection{十七}
	朝起早。早课,鲜有与。午后,生物试。余以为题优而不难。
	
	夜试,误甚。先是,余虽不能遍达,而犹能小满;而今大不善,原物理之有不得也。欲理雾霭之绪,为轻便之题。来时无多,余愿得成之。思夫先是,余能有其成,今虽小退,亦必能早归于正途。
	
	今日之事,诚无为大,唯自督之行甚。故不多言。
	\begin{flushright}
		2020年6月4日
	\end{flushright}
	
	\subsubsection{十八}
	朝起早。早课,渐觉耳得。昨日之生物优甚,欢愉之际,喜笑颜开,一头不闻。时师曰:“何以笑?”而步叱余。余顿愧然。思夫其有所得者,题之易耳。岂能语余之有所得乎?是通各科。
	
	夜试,少难于前日。
	
	余昨日言物理之不得。今始为动。理先日之误,寻得日之姿。愿能有所成。
	
	初,英语师曰:“言传身教之所阙,是以令尔等打卡。”余颇得之。是故今日令为先日,余屏书远籍,动心寻忆,终为之善。
	
	或问余曰:“务甚,何如?”余曰:“本不甚也。觉甚,辄择二三题细为之,余者可肆意。”其言者,余之先师所授也。其人有和蔼之貌,平善之态,而不拘小节。尝车载余归,虽一里之远而无所言。余明日若能为之,则平生无憾焉。
	\begin{flushright}
		2020年6月5日
	\end{flushright}
	
	\subsubsection{十九}
	明年今日,余辄就矣!
	
	今日行大礼,寻大成,终有所得。不多赘言。
	
	夜试,题易。余以不择之手,而达确言,何其幸也。
	
	明日朝,余授早课,故今少言之。自是余奋备之,俟明日,余乃具言。
	\begin{flushright}
		2020年6月8日
	\end{flushright}
	
	\subsubsection{二十}
	遑遑七日逾。先是,余以卒至之晚课,身心既乏;复加授友以题目,心外无暇。故不得为此言,阖计四日。余观先是之“俟明日”,而明日背约,是欺于公也。特先为谢。
	
	由是辄先记其事焉。九日,余有幸为课,而终究有所成者,乃先夜与俦者论议。俦者曰:“公以坦然之心,真挚之情,言语可见。而若无以限之,真情之极,便成不透。”余以为善。乃三减之。成。是故余之少功,友之盛赞,有俦者之份焉。
	
	是一周之事,日日不尽似矣。课内之务,酉初而成,余时读书耍子,皆有所乐。不时为课外之务,颇有得道。晚课毕,与友人共学,沐浴而眠。觉甚有所得。持之一周,许大得焉。
	
	今日闻明公之语,余以为可榷。鲜而明之士,正言求奇,欲以为人间之事;杂而钝之士,奋发而先,溯游而行,未尝不可得也。许阙目少臂,缓而难起速者,欲逾上述人间,时日而已。虽有人言“覆以亮剑之名,而多年之所行,不能逾”者,不能洞明耳。明公以治世之才,为万师之表,辛劳千面,求道八方,欲得水镜之业,欲成卧龙之才,是时而言不能逾者,贸乎?早乎?不宜乎?余等力为之,辄天庭之仙桃,早放苞蕾矣!
	\begin{flushright}
		2020年6月13日
	\end{flushright}
	
	
	\subsection{庆张公宇生日序}
	维遑遑疫病之时,沥沥铆雨之日,户户明灯之间,静静白纸之前,白玉请不才之身,动笨拙之笔,文贺公之生日。
	
	自余如高校以来,所遇所见,春水之甚。公者,余之初友也。虽未得共寝同眠之交,而伤感之日,拊心之时,穷困之秋,寻公可矣。张缄其口,挥止其手,其意得之。其公之心本善乎?其公之独爱我乎?愚以其友字之不忘也。而思夫余之旷野,劳烦其役,扶持之报,无以为也。公之求余者鲜,而余之劳公者多,幸甚愧甚。
	
	感激之情,一纸难书;相思之意,三页不尽。如此毕生之幸日,余无以物贻,故贻共勉之言者三:
	
	一曰尊。年力之盛,气宇之勃,余等之共有也。而长者之言,多有不曲,动吾等之心者甚。虽心近知通,亦不可有亵玩之相。此尊师之道。
	
	二曰识。余知公爱戏。而所处之地,嘈杂之言众,义理之语稀,此今之常态也。愿公不入意识之流而有独己之思。网路之人,匿名为甚,肆无忌惮。多以黯淡为真,正道为伪。此吾等之博识也。见其言而不改己道,君子也。
	
	三曰健。余等体魄,长者所重。王公年长,尚不殆动作,何况余等乎?如此陈词滥调,余等须早心领神会,不须多语。
	
	是时,余奋笔与公,情之有泄,感之有发。呜呼!余能如此言者,其唯公乎?虽暂不得俞钟之心,岂能远乎?
	
	自是,余等唯一年之余矣。莫道前路之远,莫言前路之暗。余等之路,早灭荆棘,唯余蔷薇也。愿公化于桂兰之芬,浸于芝草之芳,终究其成。诚如是,则余今日之情不空泄也!
	\begin{flushright}
		2020年5月25日
	\end{flushright}
	
	\subsection{誓于一年之期}
	夫时之速也,何人能知?由是踞余等之日,正一年矣。而是一年之行,余何以知之?唯明余之俦者,必露锋芒之齿,兼举平生之力,一冲一击,势不可当。虽有袤原之火,春江之水,难敌达理之志。欲寻尺寸以容,余先得立志。
	
	夫立志者,难立远矣。若夫其时其日,疫病再,所居闭,辄余等之可否动以先辈之行,尚难知之。况所见所得,一念之间,其易可大,何言一年?故以顺虑之,先誓言者三:
	
	一曰躬游学海。躬者,诚也。一日一试,以诚为之者,早明所不能为焉。夫不能诚者,是时发力,未尝迟也。余素不爱时力而解题,而终不动邪者,盖邪念之误,各科之衰,早自尝之。
	
	二曰助人不倦。或问以事,有所知,见而告之。所以言之者,盖初入猛虎集聚之地,余之心平而不燥;而时过境迁,冬日之凛凛,夏日之炎炎,余心易矣。乃乘此时,多为言之,则余之知得,而人亦有所获焉。
	
	三曰事求其细。人言:事无巨细,皆须慎观。余以为善。自幼为学,所以有所穷困者,皆不细也。延至今日,不改。初,余有所不得,师劝余,无用。自为题,求其细,可矣。乃曰:无所不会,唯不细耳。若为其细,必有大成。
	
	续一年之不测,许有友之聚散,亲之离合,师之褒贬,心之黯明。而无论其行,唯一之不易者,余之进也。是故余曰:无顾后,学而已。愿是六字之行,伴于余之前路,则余之大成,许可计日也。
	\begin{flushright}
			2020年6月6日
	\end{flushright}
	
	
	\subsection{画堂春·做数学难题有感}
	阴云暗涌雾连天,斧劈急悴心间。倏然九窍冒青烟,迷乱疯癫。
	
	停笔卧身闭目,凝思幽梦魂牵。欲销茫愁往那边?天外寻仙。
	\begin{flushright}
		2020年6月19日
	\end{flushright}
	
	\subsection[减字木兰花(风声诉憾)]{减字木兰花\footnote{前二联(上阕)者,白玉初到二十五班所作也。后搁置未补。疫间,白玉得之,时已无旧憾迷情,然以为其词之善,不可不补。乃作后二联(下阕)。}}
	\begin{center}
		风声诉憾,乌鸟逾天白日暗。\\
		欲遣迷情,绿亮红砖许我明。\\
		寒光煞铁,两目朱丝双耳血。\\
		莫走回廊,自是愁人愿梦长。
	\end{center}
	\hfill 2020年6月30日
	
	\subsection[沁园春·赋于最末双休]{沁园春·赋于最末双休\footnote{此日之后,白玉本级学子即入高三。高三无周末双休,故曰“最末双休”。时班主任焦子在QQ空间为同学们点赞。白玉见之而欲得之,遂作。}}
	十有一年,往事皆成,泛泛浪涛。忆初登学府,迎欢溢溢;数得捷报,喜意遥遥。窘困倏时,欣然多日,自是青年志气高。回眸望,既乘风破浪,挥剑除妖。
	
	俦公何必心焦!今天下英雄君与操\footnote{《三国志·蜀书·先主传》:“曹公从容谓先主曰:‘今天下英雄,唯使君与操耳。本初之徒,不足数也。’”}。看文章书具,颁白艳羡;师长道业,万代难超。你我之间,同心合力,智慧精灵胆气豪。行前路,愿明年今日,喜上眉梢。
	\begin{flushright}
		2020年7月5日
	\end{flushright}
	
	\subsection{贺一中高考大捷}
	南风传捷报,瑞云罩我家。学府接百岁之绩,红光耀二年之霞。眷眷父母,悉得欢欣之貌;莘莘学子,皆浸芰荷之花。精卫衔石,已填万丈之海;愚公提担,既移两山之砂。
	
	赴远俊少,离巢新雏。英雄入太宗之彀,鸿儒聚梦得之庐。明朝发达,昨日辛苦。皆夸赫赫之名,谁见劳劳之努?六艺能诵,五经已读。尝过不毛之境,已阅九重之书,兼伴一身之论,疾尽三餐之黍,笔走龙蛇之姿,身霸虎狼之浦,乃斩刀剑之棘,方散烟火之雾。苏秦聪慧,仅有悬梁之志;祖逖坚忍,只作鸡声之舞。
	
	斯人行,我俦入。将续前代之意,复植后生之树。渐入佳境,却无昭烈之志;将跃龙门,而失鲤鱼之步。骄情隐,怯意生。二年之成何论?一载之路难登。去月已逝,多负师长之意;来日方长,谁点心骨之灯?
	
	余白玉,身体不行,心智未成。一路遇贵,颇谢天地之意;三生积德,方得今日之升。愿攀书山之峭壁,渡学海之远程,弃蔽目之伪道,效佛寺之高僧。四送智慧,颇尽独身之力;一纳辛苦,少发丧怨之声。诚如是,虽毕生所愿,不得其遂,安能掩面悲啼,声振四空?
	
	呜呼!余请不才之身,动笨拙之笔,文贺吾校,兼清余志。未有足己之言,无得自矜之事,抒情而已。诚愿明年之事,吾校之能锦上添花云尔。
	\begin{flushright}
		2020年7月27日
	\end{flushright}
	
	\subsection{做导数题有感}
	\begin{center}
		亏吾今日好精神,一纸虫巢愁煞人。\\
		乱绪换得笔墨尽,劳思早教目光沉。\\
		愿为赵爽座中椅,复掸杨辉袖口尘。\\
		谜面彷徨半页里,安能竭殆五车文?
	\end{center}
	\begin{flushright}
		2020年8月11日
	\end{flushright}

	
	\subsection[望海潮(诗词歌赋)]{望海潮}
	诗词歌赋,形神散聚,文学最是销愁。函数最值,圆锥曲线,谁知酝酿秃头?抛体过圆周,钠铜铁锌铝,空腹湿眸。减数增殖,有丝分裂,苦无筹。
	
	观夫外语新仇,彼西洋乐器,特指无球。名动副形,虚拟语气,轻松哪里得求?乘上理科舟,必正心行路,无受题囚。唯愿明年此日,能喜泪横流。
	\begin{flushright}
		2020年9月4日
	\end{flushright}
	
	\subsection{永遇乐·贺楠君母难日}
	一载风霜,长君白发,升我颅际。壮雁昭捷,云霞万里,不遂蜗牛意。志存宗悫,情满昭烈,力罢书生习气。抬头望,长亭远道,凶险黯然无计。
	
	亲朋同侧,严师在野,莫叹今年清寂。却待明朝,始终成我、十二年功绩。急驰飞目,加鞭快马,此处不容停驿。搏赴梦,心头腹里,雾销雨霁。
	\begin{flushright}
		2020年9月4日
	\end{flushright}

	\subsection[西江月·雨夜不告归]{西江月·雨夜不告归\footnote{吉林市暴雨,次日休息。适毕晚课,白玉乘暴雨而归家。虽住校,未告宿管。及家,衣裤袜屡俱湿。白玉之母尝阅此词,几欲落泪。}}
	迷雾暗灯急水,墨天黏步归途。心中形役几时除?是夜销辞绝宿。
	
	空舍寂屋人去,风江雨海桴浮\footnote{《论语·公冶长》:“子曰:‘道不行,乘桴浮于海。’”}。家中谁更待吾徒?炯目粗麻旧鼠。
	\begin{flushright}
		2020年9月26日
	\end{flushright}
	
	\subsection{咏土}
	\begin{center}
		磨细到头一粒砂,那得黄卷遍天涯?\\
		往来足步力行踏,外里豆苗怒破发。\\
		无意争为煤酿户,痴心暂筑鼠郎家。\\
		明朝雨霁成泥水,笑待孤独圣女娲。
	\end{center}
	\hfill 2020年9月29日
	
	\subsection{拟书山行}
	\begin{center}
		书山何惧强劲风?学海岂顾高险峰?\\
		登山望海弱蜗意,亢力达峰鸿雁声。\\
		泛泛奇思跃纸上,湿湿老墨染襟中。\\
		胸随绣口吐真气,手与恒心植宝松。\\
		书山行,行益穷;穷还行,困几重?\\
		金榜题名无须忆,只待捷报传英雄。
	\end{center}
	\hfill 2020年11月15日
	\begin{flushleft}
		\textbf{附赵俸翊《书山行》:}
	\end{flushleft}
	\begin{center}
		新科揭榜榜前头,有人欢喜有人忧。\\
		忧者归叹欢者呼,何以墨字论赢输。\\
		状元一朝入青云,书生三秋烂翻书。\\
		仍记举夜秉烛笔,如今满纸已如空。\\
		书山行,书山行!惊回首,何晞东?\\
		可怜书生泛舟苦,只因俸薄好玉扃。
	\end{center}
	
	\subsection{贺新郎·一曲贻吾鼠}
	一曲贻吾鼠:想当初、捧侬如月,惧沾尘土。强递幼儿微挤按,瞋目呼号夺汝。尚自比、英雄父母。交友长息惟尔伴,更置于颊上轻亲抚。你不语,我心吐。
	
	不见今日七中五。但因得、明年六月,此生荣辱。夜卧寝屋人未寐,归意绵绵难阻。忖或比、行书山苦。更恐来离桑梓去,故长避离乐鸣歌鼓。空慕羡,子安骨\footnote{王勃《送杜少府之任蜀州》:“海内存知己,天涯若比邻。无为在歧路,儿女共沾巾。”}。
	\begin{flushright}
		2020年11月16日
	\end{flushright}
	
	\subsection[满江红·战雪]{满江红·战雪\footnote{化学师苏子因冰滑倒而骨折,白玉就事为之,显高三学子之意。}}
	雨雪谁来?吾在此,恭迎大驾!抬傲履,玉琼露碎,化为魂踏。定教人间阳面照,莫饶冬日贼独霸。\textbf{盐为刀,制芰荷为袍\footnote{屈原《离骚》:“制芰荷以为衣兮,集芙蓉以为裳。”},风为马\footnote{李白《梦游天姥吟留别》:“霓为衣兮风为马,云之君兮纷纷而来下。”}}。
	
	威风劲,如嚼蜡;技胜比,黔驴寡\footnote{《三戒·黔之驴》:“黔无驴,有好事者船载以入。至则无可用,放之山下。虎见之,庞然大物也,以为神,蔽林间窥之。稍出近之,慭慭然,莫相知。”}。见老夫面色,俱生惊怕。雨雪尔能今夜泯,勿迟待至三更化。尔将去,宜放手一留,男儿价。
	\begin{flushright}
		2020年11月20日
	\end{flushright}
	
	\subsection{乱写}
	\begin{center}
		千言笔,今且住,千言难买相如赋\footnote{司马相如《长门赋序》:“孝武皇帝陈皇后,时得幸,颇妒;别在长门宫, 愁闷悲思。闻蜀郡司马相如天下工为文,奉黄金百金,为相如、文君取酒,因于解悲愁之辞。而相如为文以悟主上,陈皇后复得亲幸。”}。\\
		万语口,今去酒,见酒辄灌三万斗。\\
		眼中泪,今莫坠,泪彼愈流心愈脆。\\
		耳外语,今勿取,语闻愈杂意愈沮。\\ \hspace*{\fill} \\
		
		如芒在背,六月明令难后退。\\
		如雾在天,乞来优报满心间。\\
		如砂在履,跛足骐骥何跃举?\\
		如鲠在喉,平生尽是为人愁。
	\end{center}
	\hfill 2020年12月1日
	
	\subsection{做物理难题有得}
	\begin{center}
		粗算期年事,终除题目挠。\\
		今朝报我李,早日投其桃。\\
		\textbf{学子三冬路,浮生六月桥。}\\
		愿为畎亩舜,从此代公尧。
	\end{center}
	\hfill 2020年12月11日
	
	\subsection[示友]{示友\footnote{友以伤稍休学。是日,其人在QQ空间点赞,白玉等一众友朋宣言慰之。此为白玉之言。}}
	\begin{center}
		闻道欢欣事,一拂万物哀。\\
		星繁夜影墨,珠亮目光白。\\
		地感伯牙泪,天怜幽草才\footnote{李商隐《晚晴》:“天意怜幽草,人间重晚晴。”}。\\
		孤持浊酒意,日夜望君来。
	\end{center}
	\hfill 2020年12月22日
	
	\subsection{跨年诗}
	\begin{center}
		神州今日多波霆,战疫期年速若星。\\
		学子既知学业重,名师早放名利轻。\\
		胡涂事务年关至,抖擞精神大考迎。\\
		来日重提今日事,波澜早静苦哀平。
		
	\end{center}
	\hfill 2020年12月31日
	
	
	\newpage
	\section{2021}
	\subsection{沁园春·翊公17岁自嘲以赠之}
	公汝知乎?小子今朝,意气难发。恨二年寝室,终究形役;一时疫病,再起风沙。痛扰腰肌,气填肠胃,嗟尔高三万事差!新升日,遽抛哀弃苦,笑对黄花。
	
	翊公知慧足嘉,许未见白玉号地塌。盖人间事务,浮云而已;心中笔墨,天赐繁华。勿悲平生,勿违己意,步履悠悠向梦涯。诚如是,虽竟然失意,谁唱悲笳?
	\begin{flushright}
		2021年1月17日
	\end{flushright}
	
	\subsection{百日誓师誓词}
	\begin{center}
		学子三冬路,浮生六月桥。\\
		匆匆时迫近,但余百日遥。\\
		精力满课业,笔墨润狼毫。\\
		定亮手中剑,一展志气豪。\\
		锦上添花日,不远在明朝。
		
	\end{center}
	\hfill 2021年2月25日
	
	\subsection{鹧鸪天·用山谷韵戏赋考场冷}
	芽翠枝疏春色寒,遍屋佳客泪痕干。苦居窗下披衣士,但恨难求御冷冠。
	
	温岂至,热何餐?谁来平地享清欢?便将额上存心火,烧取佳肴配酒看。
	\begin{flushright}
		2021年4月15日
	\end{flushright}
	\begin{flushleft}
		\textbf{附黄山谷原词《鹧鸪天》\footnote{此为当日三调考试语文诗歌题目。}:}
	\end{flushleft}
	
	黄菊枝头生晓寒。人生莫放酒杯干。风前横笛斜吹雨,醉里簪花倒著冠。
	
	身健在,且加餐。舞裙歌板尽清欢。黄花白发相牵挽,付与时人冷眼看。
	
	\subsection{西江月·高三赋}
	酷暑耐他湿热,坚冰堪顾严寒。高三不用泪痕干,满目星光璀璨。
	
	遍洒胸中笔墨,稍搁手里杯盘。高歌一曲破心烦,挥去额前滴汗。
	\begin{flushright}
		2021年4月25日
	\end{flushright}
	
	\subsection{咏碎毽羽}
	\begin{center}
		原来铁套聚合卿,便往人间脚上行。\\
		脚踏足踢施蛮力,羽残毽碎难哀鸣。\\
		安能为此起争语?岂可因之破浪平?\\
		纵使支离君莫弃,寄名耳上作精灵。
	\end{center}
	\hfill 2021年4月26日
	
	
	\subsection{渔家傲·拟古}
	尝见世间贫命汉,夏熬酷暑冬来颤。短发人生留欠半,终不换、梦中一碗残羹饭。
	
	肝胆相磨心底烂,一声鸡唱黄梁断。莫羡肉食空慨叹,君且看,谁居哀土心无乱?
	\begin{flushright}
		2021年4月27日
	\end{flushright}
	
	\subsection{讽彦明(其一)}
	\begin{center}
		老子今朝一匹狼,问君何日效猖狂。\\
		贪猫岂可夺龙殿,硕鼠焉能闹讲堂。\\
		有限头毛三寸短,无边风语九江长。\\
		恶言赐汝非吾意,为父但求尔作钢。\\
	\end{center}
	\hfill 2021年4月26日
	
	\subsection{小重山·用鹏举韵戏赋考场冷}
	迷路春虫草底鸣。披衣空受冷,似三更。难伸寸步况何行?烛能见,垂泪在天明。
	
	冷暖日难名。额温休迫我,跑医程。还思驻尔顷观琴,出寒曲,老子那堪听!
	\begin{flushright}
		2021年5月4日
	\end{flushright}
	\begin{flushleft}
	\textbf{附岳鹏举原词《小重山》\footnote{此为当日四调考试语文诗歌题目。}:}
	\end{flushleft}
	
	昨夜寒蛩不住鸣。惊回千里梦,已三更。起来独自绕阶行。人悄悄,帘外月胧明。
	
	白首为功名。旧山松竹老,阻归程。欲将心事付瑶琴。知音少,弦断有谁听?
	
	\subsection{讽彦明(其二)}
	\begin{center}
		早见贼儿入讲厅,问吾三晚获师经。\\
		又来猪狗涉猫事,总有乌云蔽院庭。\\
		学子朝朝望学紫,彦明日日求艳名。\\
		时难慷慨心中意,便借丹青纸上行。
	\end{center}
	\begin{flushright}
		2021年5月14日
	\end{flushright}

	\subsection{一月之期励众}
	\begin{center}
		一月笔刀露火芒,三年书剑砍张狂。\\
		勤学当趁多阳昼,早起莫贪独夜床。\\
		生为功名拼战马,要挽雕弓射天狼。\\
		待得捷报乘风至,酷暑骤间雪殿凉。
	\end{center}
	\hfill 2021年5月16日
	
	\subsection{二十五班高考临行口号}
	\begin{center}
		百年母校,赠我神通;\\
		三载师恩,助我成功!
	\end{center}
	\hfill 2021年6月5日
	
	
	\subsection{端午诗}
	\begin{center}
		屈平已作汨罗魂,楚地风俗故不闻。\\
		前日自尊龙腹子,今朝暗诩狗头人。\\
		未读经史通贤意,空把初情奉鬼神。\\
		胡扯古来学者事,但留后世满屋尘。
		
	\end{center}
	\hfill 2021年6月14日
	
	\subsection{二十五班诗}
	\begin{center}
		嗟乎吾班二十五,云风自能卧龙虎。\\
		今朝捷报南风来,来把人心敷热土。\\
		忆昔我等俱拔萃,八月十八群英会。\\
		号为亮剑誓为功,主心渐把客心坠。\\
		虽艰难遮吾等目,虽疫难盖吾等路。\\
		最喜淤泥莲蓬花,最爱寒霜松柏树。\\
		脚过六月浮生桥,回眸抬首两望遥。\\
		苦楚高三莫挂齿,英雄自在百浪淘。\\
		劝君少把酒来对,今朝之贵不足贵。\\
		待到明朝功就时,沽取新酿与君醉。
		
	\end{center}
	\hfill 2021年7月19日
	
	\subsection{吉林一中二零一八级二十五班班史精华摘录}
	\subsubsection{焦子列传选段}
	余见师长者众,然至善美者非焦子莫属。承两班之课业,领二年之繁工。性雅长志似林樟松柏,质洁本心如芰荷芙蓉。\textbf{一句褒贬,喜怒不效眉上;半边风雨,冷暖尽收腹中。解析导数,难胜深思熟虑;风物人情,育得李茂桃红。}褒词尽不足以美其意,但愿\textbf{岁月有情,莫添白发;佳音无限,来报春风。}诚如是,则本班五十二学子之情皆满焉。
	
	\subsubsection{蔡旺列传选段}
	蔡旺之为班长,盖幸事也。放豪壮之襟翼,得友朋之交谌。怀古今之雅乐,争来去之高分。出欢娱之笑语,废骄倨之伪尊。统迷蒙之学子,达班级之一身。试静思之,旺之成,非独人之力,抑亦有师生之心齐也。而心齐何故?旺之功也。是故班级之建,功劳之就,可谓正反馈也。
	
	\subsubsection{朱隽仪列传选段}
	明智而忠信,宽厚而爱人,人所求之长者焉。隽仪之行,悉皆合之。以是人之多隽仪者无惑矣。且余尝与十六班友言,友亦多之,盖其善者,一向不变者矣。孟子曰:“贤者能勿丧耳。”今朝贤者,其隽仪乎?
	
	\subsubsection{吕钊杰列传选段}
	余早闻七班有奇才者,吕钊杰也。后来二十五,以为光芒四溢,难与多言;后以伤来钊杰组,乃知其人非沉默之流,而能拥行止之德,怀治平之才,究课业而出戏语,辨真情而益同侪。\textbf{金睛火眼,来观键盘心肺;利齿伶牙,解道学问黑白。}余佩之最甚。此后余若为师者,必谈及之,届时许钊杰已成栋梁乎!
	
	\subsubsection{余璐列传选段}
	余璐者,吾本班之至交也,能与吾同游、同歌。性直诚,好敞扉,人皆知之。然璐以七尺之躯,于本班之中,洒满腹之墨,发自我之声,求生物之趣葆,望体育之功成,乃收学子之诚意,获一众之友朋。其人故也。人言:“多一友若多一路。”是故璐之将来,必有余路待之。
	
	\subsubsection{伊九源列传选段}
	余与九源,其情也深。初识,则互露本性;其后,能互问题目;疫间,约同续火花;高三,可相伴就学。\textbf{然身有顽顽不去之疾,恪守乏乏不改之事,饱经涩涩不易之苦,犹胸怀拳拳不灭之志,手握真真不变之理,肩披扬扬不疲之翅者,余之友间,唯九源是也。}如是之人,付之大业,谁复能疑?
	
	\subsubsection{王禹月列传选段}
	焦子沮然,学子悉皆察之而难措。禹月行匿名信,其略曰:“莘莘学子,今已非童稚之属也。遇遽然之风雨,虽不能吟啸而徐行,亦可不惊不动,不辱成人之身。子以学子为念,固善哉,而自身迷情,尤不可不虑。且以己身为念,流荡风声,解释心雨,则可相与共渡,虽穷冬冽风而可同往矣。”又与友朋五人购小食而贻子。子览之大振,宣言以谢。
	
	今知之而有叹焉:本班学子,所历磨难多矣。天予寒霜,地贻酷暑,内外之苦也;二载旧乏,一时新疫,前后之苦也;师者更留,学子伤病,左右之苦也。然毕业之时,合而为一,虽奥班而不能及者,何也?盖高中之凄苦,终无个人之苦,皆师生一体之苦也。师者不解学子之苦,则师生之情益远;学子不分师长之苦,则师生之恨益深,厄矣。本班学子,热心向学,品行皆善,固知所应为者,解苦、担苦、破苦也。是故往事磨难,可化轻雾云尔。
	
	\subsubsection{卓越班级·金子介绍选段}
	吾师金公迪者,长者也。抱七尺之躯,温蔼之貌,庄正之心,谈笑课堂之间,匆忙繁务之中,不失其度。范众师,渡众生。其心拳拳,其意铮铮。四送智慧,更尽独身之力;一纳辛苦,不发丧怨之声。\textbf{行歌绵绵,唱不尽绵绵情意;佳音阵阵,报不完阵阵春风}。
	
	数月已过,初战已捷。二年已逝,局促一年之遥;号角将起,不及半步之歇。
	
	
	\subsubsection{李博远列传选段}
	李博远,本班之笑果,一中之奇才。以朝暮之过往,会寒暑之去来。盈身精力,皆布共学之业;满腹真情,尽洒同窗之台。自言英俊,引起笑语欢浪;高呼令号,带出整列齐排。学子不遇博远,骤阙一份笑料;师者不遇博远,能添三分愁哀。如使博远不来本班,许学子之心,难达今日之一矣。是故余以博远者,本班之不可无者也。
	
	\subsubsection{白玉自序}
	徐白玉者,名曰照琦,字白玉。性好诗词歌赋,每有所感,则欣然命笔。初在十四,以言行多引人捧腹。既来二十五,难忘原班,但到校,辄一日数归。后渐能融入,犹出笑语,精力四溢。初在王鹤凝组,高二中,以手伤调吕钊杰组。调前调后,性好言不变。久欲为师,登台受业,多所征求。每登台,手舞足蹈,激情四射,宛若疯猴。疫间,数于组群内说文;高三之时,日归十四,假黑板受课,闻者固少,而白玉犹在台上不倦。又好歌,焦子命歌,不辞。跨年班会,任于主持。
	
	先是,白玉住校,求能自安,泯租赁之费。然虽在校,日日念家。以为思情,苦于书山之行。以是凡能归之时,白玉必归矣。高三中疫再起,校令住宿者不出,遂怒,始走读。下学期,白玉以伴父母时少,乃止住校。
	
	白玉既久自诩文人,每小事而大情。又有傲性,不在学俦,而在长者也。逢所不欣,诗文抗之,不怿遂去。在十四,恶英语字帖之行,乃作《鹊桥仙》以讽之;初来本班,满腹离愁,乃为诗《春草歌》、文《论适》以求安;为题者难,屡为赋诗,有《做椭圆大题有感》等。师生所评“其心淡然,无忧无虑”者,多以此原。
	
	白玉曰:“我在二十五凡二年,事多而不繁,友众而不俗。师长勤受课业,学子力跃龙门。满路荆棘,尽在脚下;状元桂冠,将置头前。实应文以记之,则来时回顾,琳琅在目;他人观览,收获盆满矣。况高二所行之《班级日志》已失,今之纪念,除毕业册寥寥数语外,许唯我之言也夫!”高考毕,乃为《二十五班班史》。
	
	本班师长,德才一流。虽有更易,无失大雅。焦子智美并存,邓子寓教于乐,赵子严而不厉,奚子与生共友,金子力行巨细,苏子求稳求精,李子威慈俱备,程子置生于心。作《师者列传》。
	
	蔡旺组有二班长,勤劳二年,但求无憾;计画万事,皆在周全。至于组员吕俊呈、石明鑫、刘冠群、韩金桐、谭姗姗五人,彼此偕善,整齐一体;不忘来路,追求正理。作《组别列传第一》。
	
	吕钊杰身行学委,兼任组长,合郝一鸿、王子权、周睿之明,统余璐、施苏函、曾庆洋之朗,共呼吸,同俯仰。乃至声名大成,心胸开敞。二年和睦,他人难仿。作《组别列传第二》。
	
	马赫咛行在组长,万事无忧;组员七人,各有千秋。侯冠宇、王可欣、季珈铭行课代表事,尽职尽责;李明基、陈姿宜、姚昕航各领本务,在严在苛。共王可欣作《组别列传第三》。
	
	王方佟达厚意,表深情,爱人爱己,危言危行。六班五人,融于十二;复读一人,归于正名。乃结金泓旭、姜玥含、李小来、沈佳馥、盛鑫、任心童、丛子贺为一。作《组别列传第四》。
	
	王鹤凝心向学业,性犹谦和。组凡八人,不用多责。赵家宇以速闻,于丹以友闻,王姝力以德行闻,段雨含、李星燃、林义贺以勤而不倦闻,赵鹏飞以爽闻。作《组别列传第五》。
	
	孔祥宇兴趣广,智慧足。与组员伊九源、孙铭瑶、王怡然、张宇、张羽赫、曲政旭凡七人,皆为高士,无所畏服。能若诸葛,才似相如。泯悲哀于苦事,达德行于修竹。作《组别列传第六》。
	
	刘伊涵待人以善,克己有方。个人德满,整组谐康。兰思达多出奇问,王禹月多出文绩,闫英祺多出善举,高浩天多出妙想,王琳琈多出优解,李博远多出笑语。作《组别列传第七》。
	
	维本事已叙,而末事未言,总以为阙。文科师长四人,皆爱学子;年级组师长无人,俱有所记;去者三人,悉有所成;复有班中留言板,弥足珍贵。作《补事叙第一》。
	
	维初成前补,犹有阙者。高一之事,原班记之而望本班;杂杂往事,学子存之而有厚情;二级结论,本班假之而有佳绩。作《补事叙第二》。
	
	维既成班史,来者视之,不确其时,则史无用矣。作《大事纪》。
	
	维赵子之所重者,打卡也。本班英语之成者,打卡也。留以为忆,其意重也。托吕钊杰作《课外打卡》。
	
	维疫间焦子命为“卓越班级”,一日而成,文辞皆善。然则所录之校刊终未尝见,余恐文有所失,作《卓越班级》。
	
	维我来度二年,未以为足。师生之情,固留回味;同窗之好,尽在心中。愿使此身此躯,再坐堂上;持纸持笔,来答理综。纵观古今,似我之浓浓不舍者多矣,然如是之人,俱有以记之,我复何所为耶?盖仿古太史公之意,为史为文,造言造书而已。思之信之,乃能为班史也。作《白玉自序》。后世人见之,许能知我之情意矣。
	
	白玉曰:我记吉林市第一中学二〇一八级二十五班之班史,凡十四卷。未有违心之论,尽是肺腑之言。
	
	\subsection{沁园春·临别赠王姝力}
	那里牢骚,换取三年,体外诉发。共推心换意,未多用酒;品书论道,正少斟茶。昨日欢欣,明朝离苦,与尔征程万路达。腔中志,召吾侪明探,前道风沙。 
	
	平生满腹清嘉,莫应为悲酸唱地塌。盖人间事务,浮云而已;胸中笔墨,天赐繁华。勿哀盈亏,勿违己意,学海方舟向对涯。云霄外,但君能平定,千里浪花。
	\begin{flushright}
		2021年8月9日
	\end{flushright}
	
	\subsection{酒后乱写}
	\begin{center}
		老子饮酒,饮出名堂,甚是欣慰。\\
		推杯论盏,那里顾得,今朝酒贵。\\
		灵光屡现,诳语频出,与谁同醉?\\
		乃知饮者,古来留名,人之所谓。\\
		今夜榻上,大字向天,伴鼠而睡。
	\end{center}
	\hfill 2021年8月24日
	
	
	\subsection{贺新郎·二曲贻吾鼠}
	二曲贻吾鼠:望三年、寒窗相伴,已成尘土。俄顷少年离家去,不舍风情与汝。心遂领、别时父母。从此难求知我者,有牢骚固是无人抚。纵满腹,向谁吐?
	
	与君度日五千五。自拙薄、渐焉明晓,男儿荣辱。况乃别离今生事,早晚非由吾阻。应止用、长凄短苦。休取悲凉箫歌意,要月台为我鸣锣鼓。教尔见,丈夫骨。
	\begin{flushright}
		2021年9月2日
	\end{flushright}
	
	\subsection{自家徒步至一中记}
	余念旧,知我者皆会之。以是故人之来,无往不欣,乃尝见欺,犹不以为不得。余家去一中凡七公里,弗远弗近,又惜租屋费贵,遂择住校。然则日日恋家,其情难绝。高三下而止之,日夜通勤,早出晚归,日销三十余钱。余固知挥霍,然此情不泄,余中终有所阻。身在车上,目向窗外,时往意流,逐起心意,欲一徒步而往焉。
	
	而今业结,即就远行。桃李相遇,一中三年,已作去日矣。当日之欲,徒步之事,尚未尝达矣。明日乘车,再见故乡,盖冬日也矣。是故此朝心决,七里之途,往观母校矣。
	
	是日也,金日洒地,暑气未灭。余顶新帽,自延安路入吉林大街,又步松江中路,乃至学府路。途中口渴,所携既尽,而傍江无店,故久而未饮。行三二,方遇一店。祛渴而行,遂至一中。自外观之,其楼犹立,其塔犹威。自是心满意足,轻歌而归。
	
	同行者:周睿。
	\begin{flushright}
		2021年9月3日
	\end{flushright}

	\subsection[游子碎碎念]{游子碎碎念\footnote{写于东北限电之时,有借鉴罗大佑《未来的主人翁》。}}
	\begin{center}
		我穿过红色砖瓦的楼区看着西装革履的人\\
		我走在夜幕下的跑道上面裹进太阳般的路灯\\
		一辆辆黄色的单车驶入我温暖的家乡梦\\
		曾经一度我认为这两个地方没有什么不同\\ \hspace*{\fill} \\
		
		风声在我耳边不断地唱着一曲悲伤的歌\\
		恍惚间我感觉自己是个卑鄙的异乡客\\
		那悲欢离合的故事怎么能全部对游子说\\
		在黑暗而寒冷的深夜里品味古老的生活\\ \hspace*{\fill} \\
		
		每一个从黑土地里出来的游子\\
		焦急而饱含深情地念着故乡的名字\\
		每一个黑土地里的灵魂在沉思\\
		今天的好运是否来自他们昨天的恩赐\\ \hspace*{\fill} \\
		
		每一个离家的游子都怀着一颗面向故乡的心\\
		我听到他们的呐喊在这里与千里之外统一\\
		我们不要蜡烛的微光变成黑夜的火炬\\
		我们不要市郊的泪水洒遍关东的田地\\ \hspace*{\fill} \\
		
		每一个从黑土地里出来的游子\\
		焦急而饱含深情地念着故乡的名字\\
		每一个黑土地里的灵魂在沉思\\
		今天的好运是否来自他们昨天的恩赐\\ \hspace*{\fill} \\
		
		没有人会希望他听到的只是粉饰思想的故事\\
		没有人会愿意住在一个昏天黑地的街区\\
		没有人会想要一个提心吊胆消极的假期\\
		既然如此,请别再问我们是否要回家去\\
		回不回去,我们还回不回去\\
		回不回去,我们还回不回去\\
		回不回去,我们还回不回去\\
		回不回去,我们还回不回去\\
		回不回去,我们还回不回去\\ 
		……\\\hspace*{\fill} \\
		
		我们不要蜡烛的微光变成黑夜的火炬\\
		我们不要昨日的欢笑变成明日的叹息\\
		我们不要被那资本的游戏夺去发言的权利\\
		我们不要被那网络的言语占据我们的思绪\\
		我们不要市郊的泪水洒遍关东的田地\\
		我们不要灯火通明的世界遮蔽黑暗的屋宇
	\end{center}
	\hfill 2021年9月26日
	
	
	\subsection{望海潮·物理好难}
	金公足下:游生白玉,今遭物理风沙。当忆考前,何期此日,嗟余秀发堪抓!眉作欲连八,腹鸣退堂鼓,无意秋花。乍起凉风,遽出枯月,响寒鸦。
	
	难逾质点坑洼,诧积分力矩,矢量乘加。刚体定轴,时空可变,一心万道创疤。辛苦换红叉,兴味成云雾,地动天塌。书尽牢骚起笔,空笑个哈哈。
	\begin{flushright}
		2021年11月5日
	\end{flushright}
	
	\subsection{贺新郎·三曲贻吾鼠}
	三曲贻吾鼠:梦昨宵,欣然舐我,耳鼻灰土。乍醒知疑君感召,狂笔缘由诉汝。且莫怪、叨儿絮母。月照人生千万里,算蛾眉只有凉风抚。伤远去,恨无吐。
	
	今天是个星期五。欲平心,难出难入,踏得屈辱。别取离人情勿捡,自是一腔隔阻。犹尚愿,厚积人苦。迫考未尝激奋志,厌风雨乱韵杂锣鼓。鬼扯话,响听骨。
	\begin{flushright}
		2021年11月19日
	\end{flushright}
	
	\subsection{随手写个小绝句}
	\begin{center}
		京城没有二三事,\\
		师大莫得廿五人。\\
		料定此心非我有,\\
		空留千里待家春。
	\end{center}
	\hfill 2021年11月23日
	
	\subsection[安提莫尼(Antimony)一世]{安提莫尼(Antimony)一世\footnote{讽彦明,有借鉴罗大佑《绿色恐怖分子》;Antimony,指化学元素“锑”。}}
	\begin{center}
		你是安提莫尼安提莫尼一世\\
		侵占周六给人讲鬼故事\\
		你让大小老师对你垂头听旨\\
		安提莫尼形象跃然于纸\\
		你要营造美丽校园内外物质\\
		赶走校门口的商品街市\\
		你在高三楼的廊中徘徊观视\\
		嘴里揣着什么样的历史\\
		您是内卷浪潮中的天使\\
		开辟强制自由的端始\\
		您来赐予这个学校价值\\
		高考状元光照吉林市\\
		你要用你九成新的脑袋瓜子\\
		迎接每一届的新生蛋子\\
		我们将会永远歌颂你的故事\\
		吉林一中安提莫尼一世\\
		吉林一中安提莫尼一世!
	\end{center}
	\hfill 2021年12月13日
	
	
	
	\subsection{跨年写现象}
	\begin{center}
		二〇二一说了一声拜拜\\
		我们把悬着的心放了下来\\
		那些人世间不能抒发出的澎湃\\
		就放到二〇二二继续感慨\\
		有人在艰辛整理一点回忆\\
		有人要剖析证伪一点真理\\
		有人在高数题海里面闲逛\\
		有人把论文写在大地上\\
		有人被逼无奈改变生计\\
		有人在中风险区自我隔离\\
		有人在因特世界当众键盘格斗\\
		有人把一生积蓄一朝骗走\\
		有人在黑夜之中内卷玩个没够\\
		有人在大白天里到处梦游\footnote{有借鉴罗大佑《现象》。}
		
	\end{center}
	\begin{flushright}
		2021年12月31日
	\end{flushright}
	
	\newpage
	\section{2022}
	
	\subsection{一中赋}
	\begin{small}
		序:维白玉离家乡四月,久怀笼鸟之思,常抱郁凄之意。今归,无所事事,闲览后生之所成,同窗之所录,心绪起而欲发,嗟叹出而难入。复观考前师送学子所留,见一联曰:“少年此去自是春风十里,豪杰归来莫忘一中三年”,其词也易,而乃引涕泗徐下。适阅《昭明文选》,仰大贤之风,乃决以拙笔为母校作《一中赋》,以考前最末日展焉。倘后来有思,则能阅此赋以为慰也。
	\end{small}
	
	子居课上,曰:“嗟乎二三子,三年何其速也!君之初至,踏二十八顷之一角,观草木兮明丽,点青石兮彷徨。浸学子之意气,浴丁香之芬芳。幼鳞入水,饱吸一鉴之气;白鹄腾空,骄映海霞之光。校史之辉也灿,魁星之体也庄。以是观春叶,眇夏阳,抚秋水,对冬霜。原道理以求索,列操队而成行。谊同道之朋辈,餐异烹之牛羊。期年也,高三之事制汝,乃至限于数罟,囿于杂务,朝往暮归,不知所如。而今大考将至,三年之事,云烟而已。分离之事,今日而已。不知君自此而去,可秉我‘敦品励行,热心向学’哉?白玉,尔何如?”
	
	白玉立而扬言曰:“呜呼!学子三冬路,浮生六月桥。龙门咫尺近,终果百年遥。前日大捷,手承佳绩之笔;今朝雏鸟,身去母校之巢。愿负昨日之妙笔,作明日之狼毫,击书山之博浪,讨学海之野蛟。勤修课业,毋忘师长之语;砥砺德行,争为后生之标。子可待:游鱼浮兮行鸽飞,豪杰归兮伴芳菲!”众闻之,齐颔曰:“诚如白玉之言。”
	
	子欣然而笑,曰:“二三子之去也,皆抱琼花之得,各享诸葛之遇;我之留也,寓茫茫之深情,接赫赫之伟绩。承蒙汝等之厚,在此谢焉。”乃躬。众皆拊掌。
	
	俄而外,以雨稍阻于口,顷出。列双队,起横幅。楼染青天之色,云停初雨之湖。懿德光照,班牌享幸;美人手捧,鲜花蒙福。红颜佳貌,显是盈面欢喜;青筋义胆,自有满腹诗书。缓行焉,踱步焉。逾宇焉,望坪焉。行伍从容,精神扬乎眉上;后生夹道,旗海涌乎天边。呼兄姊以鼓舞,效稚子而争先。事遽然无所措,遂拱手于胸前。
	
	于是风乍起,雨骤落,伞为阻,衣作隔。乃提步疾行,入北门门口之棚。两旁之士,皆长者也。暂褪帽收伞,将献花谢恩。礼虽简而意重,行虽朴而情深。忽见棚外一人,雨中屹立,身上淋迹似画,鬓间白发如针,声震如雷鸣仲夏,拳起若苗破初春,但闻口中出“吉林一中”,众师和曰:“高考必胜!”往来数次,以致龙王惊骇,泼晴止雨;行鸽惧动,飞魄失魂。
	既出,队骤散,如沙之入海,积而难聚。然其蓝白之色,浸乎墨绿之间,盖乍明也。
	
	白玉登车,久顾不止。
	\begin{flushright}
		2022年1月23日
	\end{flushright}
	
	\subsection[致高三]{致高三\footnote{有借鉴罗大佑《家II》。}}
	\begin{center}
		每一盏前行的灯\ 点亮混沌的双眼\\
		每一株待放的花\ 等待暑日阳光的恩典\\
		每一本厚重的书\ 漆上希望的墨点\\
		每一扇开启的门\ 见证你那最真的笑脸\\
		愿你有温暖的家庭\ 还有同窗师生的真情\\
		让你那疲惫的心灵\ 没有恐惧孤单的阴影\\ \hspace*{\fill} \\
		
		数一数屋里的人\ 看看远近的风景\\
		暖一暖冰冷的桌\ 听听伙伴无心的叮咛\\
		转动你昏沉的头\ 莫要搞错了终点\\
		走过那拥挤的桥\ 让我眼见你潇洒的脸\\
		愿你有未来的憧憬\ 还有脚踏实地的清醒\\
		让你那纯洁的心灵\ 有着火焰一般的热情\\ \hspace*{\fill} \\
		
		三年之前满怀兴奋的心情你走到这屋檐下\\
		而今你将要飞向那广阔的天空告别留芳华\\ \hspace*{\fill} \\
		
		每一盏昏黄的灯\ 照亮可爱的睡颜\\
		每一株盛开的花\ 实现五彩缤纷的诺言\footnote{罗大佑《东方之珠》:“东方之珠,整夜未眠,守着沧海桑田变幻的诺言。”}\\
		每一本厚重的书\ 永存希望的墨点\\
		每一扇开启的门\ 铭记你那最真的笑脸
		
	\end{center}
	\begin{flushright}
		2022年3月23日
	\end{flushright}
	
	\subsection{祖先}
	\begin{center}
		如果你是祖先\ 何妨打开你面前的门\\
		把身影投在银幕\ 看看你沉默的子孙\\
		他们或许早已忘记你那无畏的灵魂\\
		忘记你徒手扼住你命运的巨轮\\ \hspace*{\fill} \\
		
		如果你是祖先\ 何妨走近你面前的人\\
		轻轻伸出你双手\ 抚摸他身上的创痕\\
		他无从表达自己心中那神秘的苦闷\\
		只好把冬夜当作是春天的清晨\\ \hspace*{\fill} \\
		
		你的大胆\ 你的勇敢\\
		遗失在时光中一去不复返\\
		你的真诚\ 你的热情\\
		被麻木所替换而不知踪影\\
		你的善良\ 你的坚强\\
		早已经不是任何人的榜样\\
		你所存在的只是一个名字\\
		你所留下的只是一段故事\\
		可叹的名字 悲哀的故事\\ \hspace*{\fill} \\
		
		如果我是祖先\ 我愿做一个朴素的人\\
		把身体扎根土壤\ 抛去那不实的名尊\\
		用最鄙夷的眼神去藐视黑白的纠纷\\
		用最诚恳的拥抱感谢父母的恩\\ \hspace*{\fill} \\
		
		如果我是祖先\ 我愿关上我后来的门\\
		让世界减速运转\ 便利我未来的子孙\\
		当我们的话语成为抨击心灵的美文\\
		你便能感应我犹存的不屈的魂
	\end{center}
	\hfill 2022年3月26日
	
	\subsection[民科雷绍武]{民科雷绍武\footnote{讽“民科”雷绍武,有借鉴罗大佑《弹唱词》。}}
	\begin{center}
		手指勾一勾\ \ 崭新的知识\\
		贴吧兜一兜\ \ 咏雷的故事\\
		转头溜一溜\ \ 井盖做帽子\\
		米饭扣一扣\ \ 小麦的卫士\\
		嘿呦嗯嘿呦\ \ 天地的卫士\\
		嘿呦嗯嘿呦\ \ 天地的卫士\\ \hspace*{\fill} \\
		
		人读半生书\ \ 难知半生路\\
		手握真理书\ \ 哪能便自负\\
		无知是牛顿\ \ 无德是楼主\\
		无耻是官科\ \ 无赖是绍武\\
		嘿呦嗯嘿呦\ \ 无赖的绍武\\
		嘿呦嗯嘿呦\ \ 无赖的绍武\\ \hspace*{\fill} \\
		
		人在世间生\ \ 谁不归尘土\\
		白发一头新\ \ 尝尽岁月苦\\
		今朝是老人\ \ 曾经是父母\\
		无聊的面孔\ \ 年轻的动物\\
		嘿呦嗯嘿呦\ \ 年轻的动物\\
		嘿呦嗯嘿呦\ \ 年轻的动物\\ \hspace*{\fill} \\
		
		人在生命中\ \ 何处知音求\\
		理论换不得\ \ 只来嘲笑口\\
		有家空似无\ \ 无朋却若有\\
		伪装的癫疯\ \ 哗众的小丑\\
		嘿呦嗯嘿呦\ \ 民科的小丑\\
		嘿呦嗯嘿呦\ \ 民科的小丑\\ \hspace*{\fill} \\
		民科的小丑!
	\end{center}
	\hfill 2022年4月10日
	
	
	\subsection[写给成年的自己]{写给成年的自己\footnote{有借鉴罗大佑《啊,停不住的爱人》。}}
	\begin{center}
		啊,成年就在目前\\
		告别了年少无拘无束的稚嫩\\
		悲哀欢欣冷漠暂且不计\\
		要等待遣散无尽黑暗的清晨\\
		红灯黑瓦承包得了城市\\
		碰不到黄河蓝天白云间的我们\\ \hspace*{\fill} \\
		
		啊,成年就在目前\\
		我依然痛恨人间反复的冰雪\\
		生而为人已然注定不易\\
		却还要直面铺天盖地的扳机\\
		人要注意识别精神毒剂\\
		黑白的故事要靠英雄们去努力\\ \hspace*{\fill} \\
		
		让我这松弛的肩膀\\
		扛起那点亮未知幸福的扁担\\
		轻轻整理邋遢的容貌\\
		寻求那能够慰藉疲惫心灵的臂弯\\ \hspace*{\fill} \\
		
		啊,成年就在目前\\
		如果我遗忘当初承诺的誓言\\
		浑浑噩噩原地辗转来回\\
		请岁月给我无边无际的后悔
	\end{center}
	\hfill 2022年4月27日
	
	
	\subsection{为数学师赋励高考学子(其一)}
	\begin{center}
		殷勤苦难三冬长,高考将临学子忙。\\
		三角集合铺手路,解析导数问心房。\\
		题多练就从容笔,时少偷寻暗夜床。\\
		待到明朝捷报至,雪飞炎海变清凉。
	\end{center}
	\hfill 2022年5月14日
	
	\subsection{为数学师赋励高考学子(其二)}
	\begin{center}
		学子三冬路,浮生六月桥。\\
		匆匆试炼毕,龙门咫尺遥。\\
		忆昔双度疫,风雨布江潮。\\
		面寄荧光幕,身托灯火宵。\\
		真龙生淡水,好马起肥膘。\\
		人惰山河贬,人勤天地褒。\\
		我等数学事,人中觅凤貂。\\
		于君传道理,求取栋梁高。\\
		解析留目眩,导数闹心焦。\\
		苦难无多日,掌间亮刃刀。\\
		英雄今盖世,胆气破云霄。\\
		待到乾坤定,名成后世标。
		
	\end{center}
	\hfill 2022年5月15日
	
	\subsection{为数学师赋励高考学子(半作)}
	\begin{center}
		每一段初始的轴\ \ 铺开解析的画卷\\
		每一张立体的图\ \ 说着角度推究的寓言\\
		每一次求导之后\ \ 我们渴望它单减\\
		每一个定理之中\ \ 蕴藏几多苦难与甘甜
		
	\end{center}
	\hfill 2022年5月15日
	
	\subsection{咏长征}
	\begin{center}
		扎根乍破危如缕,遂北求方登路途。\\
		三面军衣尘作线,两年兵海苦为书。\\
		艰难涂就红旗墨,霜雪化得金武符。\\
		待到明朝春晓日,新发革命扫贼污。
	\end{center}
	\hfill 2022年5月26日
	
	\subsection{代远野作}
	\begin{center}
		提身向野兽,宝剑腥难安。\\
		绛绛新茶色,皑皑旧榻斑。\\
		一朝先辈客,万代老基杆。\\
		何日被雷普,重寻一世欢?
	\end{center}
	\hfill 2022年5月27日
	
	\subsection{网络拳师}
	\begin{center}
		键盘敲万里,枯木带双枝。\\
		好骂起裆客,勤拥钟乳石。\\
		笔提云雾重,拳落眼眸湿。\\
		数载身名灭,犹为后世师。
	\end{center}
	\hfill 2022年5月27日
	
	\subsection{南乡子·重下智学网有感}
	学海岸洲村,寒路三冬恨汝深。落笔青春今不管,昏昏。解释轻枷放浪身。
	
	游鸟念巢温,假意真情苦难分。赤色原来人作注,谆谆。无忘平生去日恩。
	\begin{flushright}
		2022年6月7日
	\end{flushright}
	
	\subsection[减字木兰花(今宵约尔)]{减字木兰花}
	\begin{center}
		今宵约尔,血色罗裙珠翠耳。\\
		共赏黄昏,碎步轻声待酒温。\\
		瓜田夏野,脉脉情思颜上写。\\
		入挽新楼,荡尽平生无复求。
	\end{center}
	\hfill 2022年6月10日
	
	\subsection[清平乐(天边云换)]{清平乐}
	天边云换,星汉连宵淡。灯火新烧芦苇岸,白刃割裁双段。
	
	新情旧况难书,何存笔墨工夫?梦里日兴雾尽,手中兔死人哭。
	\begin{flushright}
		2022年7月10日
	\end{flushright}
	
	\subsection[临江仙(愤愤难收天际志)]{临江仙}
	愤愤难收天际志,手中多见途穷。但求鲜血去时红,灵愁无限意,慷慨付英雄。
	
	忧恨休归无用事,\textbf{长成不悔儿童}。假思救苦少年空,对床眉蔽眼,屋外大江隆\footnote{黄庭坚《王充道送水仙花五十枝欣然会心为之作咏》:“坐对真成被花恼,出门一笑大江隆。”}。
	\begin{flushright}
		2022年7月10日
	\end{flushright}
	
	\subsection[风霜]{风霜\footnote{写新冠疫情。}}
	\begin{center}
		能被放在一起\ \ 是不幸\ \ 我清醒\\
		可是依附的阴影\ \ 却逐渐\ \ 成为你我的屋顶\\
		有谁人能够知晓\ \ 患难过后\ \  可否残留真感情\\
		两年的风霜\ \ 像绚烂的流星\\ \hspace*{\fill} \\
		
		转眼秋冬又春夏\ \ 谁料风雪续风沙\\
		或许这阴影\ \  爱着我\ \  也是少年的情感家\\
		命运是脱缰的野马\ \  我想是沙洲徘徊的野鸭\\
		解构我的生命\ \ 看着你的生灵\ \ 总忘不了这段年华
	\end{center}
	\hfill 2022年7月15日
	
	\subsection{咏墙}
	\begin{center}
		砌者应知人断肠,守拙内外俱悲凉。\\
		内无雨雪新朝短,外有风霜老夜长。\\
		人问何时再呐喊,我答今夜但彷徨。\\
		天涯芳草佳人问,换取英雄泪几行?\footnote{辛弃疾《水龙吟·登建康赏心亭》:“倩何人唤取、红巾翠袖,揾英雄泪?”}
		
	\end{center}
	\hfill 2022年7月18日
	
	\subsection[满江红·重写赠金子诗]{满江红·重写赠金子诗\footnote{初,吉林一中校报一栏目曰“卓越班级”,盖录班级优异者于上,有班级介绍、班花介绍、各科师者介绍等。白玉为写“卓越班级”金子部分,语见6.14.8。尝有《满江红》词,因时间不足,遂直用2.3《伪满江红》。时白玉修订班史,阅之,以为大恨。遂重写其词。}}
	主任双年,添足了、鬓间霜月。书几遍、力学行止,电流阻越。人事从来劳役体,独情多自闹长夜。回眸望,念二载同舟,如何谢!
	
	民生愿,难泯灭;学子梦,犹真切。渡征途久矣,必然明确。书本苛求纸背透,\textbf{闲时但向梦中借}。待明朝,锦上添花时,迎\footnote{初,此字为“活”,后改为“燃”、“留”,后整句又改为“君生靥”。后赵俸翊曰:“‘迎’字佳。”遂定为此字。}君靥。
	\begin{flushright}
		2022年7月20日
	\end{flushright}
	
	\subsection{西江月·临走心不宁作}
	今夜惊雷窗外,明朝暗雨心窝。秋风夏日荡春波,痛与风云皆热。
	
	兔问回程年月,鼠曰不走如何。苦将回忆闹搜罗,大字难眠是我。
	\begin{flushright}
		2022年8月26日
	\end{flushright}
	
	
	\subsection[中秋命题]{中秋命题\footnote{此年中秋节即为教师节,会上学期期末考试延考。白玉以此赠大学班主任申老师。}}
	\begin{center}
		人间硕果中秋盈,晴夜西风若有情。\\
		鱼米肴携入胃海,月灯窗赏绘睛屏。\\
		书生内旨徐除诞,师长伦侪慨赋灵。\\
		旧试新期学次日,小才老理妙成英。
	\end{center}
	\hfill 2022年9月10日
	
	\subsection{游颐和园记}
	夫人之性也,逆于贵秀,怡于幽清。身栖局促,无那一时之便;神游山川,终究万世之情。是日也,日高区澈,云散风轻。秋景效春初之丽,暖流销冷气之凌。于是约朋延友,计路规营,乃乘公车,往颐和园一游。朝出,至暮得餐而归。
	
	是园者袤也。环而游之,长约十里;遍路而行之,许二十里不能尽也。东门巍巍,先露“涵虚”之义;绿湖点点,若存“昆明”之灵。彩舟浪越,修塔宾迎。黄山留翠,赤槛泛青。狮守门前,汹汹吼龙雀之固所;联书双侧,俨俨照枫柳之落英。绵绵水道,悠悠廊亭。鱼乐粼粼之波,鸟游暖暖之汀。大桥虹立,承凭十七拱孔;小岛巢盘,转拜六面公卿。毫馆精藏,琳琅十代之品;戏台高矗,婉转百年之莺。
	
	维人有悲喜,时有兴衰。后人新履,前代旧牌。谢堂虫燕,终入万姓之户;王府日月,亦照百家之台。以是逝者之功,往事之德,彼所贻赠,如是焉耳矣。后人鉴之,盖持千年之智慧,追一代之去来,觅阳春之美景,寻大块之妙才,赏风流之遗业,品世道之黑白,收落叶以游戏,思道理而徘徊。如是则人不必戚戚然于世,慊慊然于人,折骨伤身,以至垂命矣。愿后人思之,深意在其中也。
	
	同游者:云南吕磊、贵州林杰、湖北张梓峰三者也。
	\begin{flushright}
		2022年11月5日
	\end{flushright}
	
	\subsection{师}
	\begin{center}
		师。\\
		导数,唐诗。\\
		餐笔墨,种兰芝。\\
		人情冷暖,家国驱驰。\\
		鸡声朝日里,灯火暮钟时。\\
		有念光阴似箭,无忘恩威并施。\\
		贤圣古今传玉业,鸟雏风雨送巢枝。
		
	\end{center}
	\hfill 2022年11月21日
	\begin{flushleft}
		\textbf{附余昕泽原诗(节选)如下:}
	\end{flushleft}
	\begin{center}
		师。\\
		匠心,情挚。\\
		染墨香,蕴悠辞。\\
		教泽尽处,尺素难织。\\
		先生辛劳驻,桃李手中卮。\\
		诵罢春夜喜雨,重吟己亥杂诗。\\
		人生得意半山故,恩师教诲莫别离。
	\end{center}
	
	\subsection[沁园春·代人赋]{沁园春·代人赋\footnote{为学院一二九合唱作朗诵词,仿田汉《毕业歌》之义。以疫故,终不用。}}
	国事盈天,病虎潜龙,累卵苍黄。恨墙边风雨,不拦忧患;城中名姓,皆负痍疮。草木垂头,山河挥泪,嗟我中华万事伤。学人问:此天高地阔,谁主兴亡?
	
	千年内外风霜,料难过青年血脉张!念浓浓热血,应流壮志;泱泱华夏,必聚优梁。砖瓦行宫,泥沙瀚海,数万英雄浩气长。同诸位,做滔滔巨浪,冲碎朝阳!
	\begin{flushright}
		2022年11月22日
	\end{flushright}

	
	
	\subsection[遣怀七首]{遣怀七首\footnote{结构仿写曹植《赠白马王彪》。}}
	\subsubsection{一}
	西风刺人愿,年末张苦辛。红瓦蒙都邑,青云播祸荫。倦隔灰银幕,笑裹蓝白巾\footnote{口罩。}。年驰见景老,时锐伴身新。鸦鸟起寒树,往复有所因。叶黄垂冷地,惭愧世上宾。徘徊棉衣外,举目非我亲。借问英雄事,如愿谁伤心?
	\subsubsection{二}
	伤心无泪流,染面风也愁。出门望东路,旷旷无人游。茫茫前步隐,绒屐抹踝油。陡陡砂石怪,车船焉可求?掩盖翳云木,踌躇踏土抔。人陌无穷变,狗窦万道由。悲苦修吾发,辗转或秃头。汹涌此怀溢,强风似轻柔。
	\subsubsection{三}
	轻柔耳不闻,道引古诗存。人心如妆绘,白粉何乎纯?三月终期见,冢课枯骨魂。老理难应手,新学犹致昏。惯看书中士,砥砺在乾坤。智慧充饥禄,风骨妙绝伦。彭祖谁堪比?孔丘谁谒门?且饮荔枝酒,慰我淡精神。
	\subsubsection{四}
	精神将何足?犹思归我庐。牛刀宰幼雉,野火灭离蜈。父母勤业作,儿女苦书读。纵有万舟水,难收天下儒。虚长十八岁,嫩颊未蓄胡。激愤明道理,能知非我途。他日凌云遂,游戏山与湖。休言无巢恋,此巢不可居。
	\subsubsection{五}
	可居岂可安?往来飘云端。夜半惊白鬼,晨曦怕简餐。宝马\footnote{谐音,“码”。}难过夜,辛勤未容宽。时序南墙堵,消息北屋搬。楼宇门前树,独迎饭菜甘。风声从此诉,摇动笑憨憨。光阴顷刻去,无再少年欢。念此先道顾,塌然肺心酸。
	\subsubsection{六}
	心酸有所规,天地明翠微。道理随思悟,来去苦不推。太平因粉饰,优劣在思维。从来浮萍客,难逃老金催。人生浑如誓,昨日与今非。情况瞬息去,安闲那可追?黄发不可预,馑饭若春晖。不如伴美酒,食尽首阳薇。
	\subsubsection{七}
	呜呼归去来!斟酌动我怀。文章时同度,感沛久兴裁。野土生穷岁,曲蘖秋日埋。井田一何固,芷兰不可栽。昨日初冬雨,凝洁荡无埃。冷空凌云碧,寒木弃草衰。料知回程路,此心卑且乖。笔尽情无赦,何言慰我哀?
	\begin{flushright}
		2022年11月25日
	\end{flushright}
	
	\subsection[念奴娇(人生途次)]{念奴娇}
	人生途次,数愁云黯雨,地贻天赐。万里难求神鬼梦,跬步芝屋鱼肆\footnote{《说苑·杂言》:“与善人居,如入兰芷之室,久而不闻其香,则与之化矣;与恶人居,如入鲍鱼之肆,久而不闻其臭,亦与之化矣。”}。风吼层林,豹栖巢野,仁义无今视。颜回难老,草抔烟火易逝。
	
	我兔我鼠闻之,咍然相慰:“贤圣犹君是。长路当行如见碍,\textbf{笔墨旦夕流泗}。逝者无期,此身有尽,何必痴衢世?”愿家虚土,久临宣纸文字。
	\begin{flushright}
		2022年12月12日
	\end{flushright}
	
	\newpage
	\section{2023}
	\subsection[一剪梅(新雪昨宵檐下穿)]{一剪梅}
	新雪昨宵檐下穿。风也姗姗,泪却潸潸。零落生活守律难。慢醒三竿,快饭双餐。
	
	休假时常心溢澜。睿早得欢,璐又脱单。愿试身着野兽衫。人也弯弯,月也弯弯。
	\begin{flushright}
		2023年1月20日
	\end{flushright}
	
	\subsection[致高三II]{致高三II\footnote{有借鉴王武雄《宝贝心肝》。白玉对此作不满,故未尝发布。}}
	\begin{center}
		总会有人\ \ 说你该\ \ 把握好时代\\
		总会有人\ \ 当你明珠\ \ 捧你在胸怀\\
		也曾晚寐\  \ 也曾有愧\ \ 也曾问未来\\
		也曾落泪\ \ 黑圈眼中\ \ 书本滥如灾\\ \hspace*{\fill} \\
		
		船帆也重\ \ 风雨声\ \ 红衣\footnote{明·陈耀文《天中记》卷三十八引《侯鲭录》:“欧阳修知贡举日,每遇考试卷,坐后常觉一朱衣人时复点头,然后其文入格。……因语其事于同列,为之三叹。尝有句云:‘唯愿朱衣一点头。’”}来驻看\\
		函数果蝇\ \ 分子恒星\ \ 理科的梦幻\\
		看你手算\ \ 看你苦推\ \ 看你图上画\\
		看你背诵\ \ 看你补抄\ \ 朋友的作答\\ \hspace*{\fill} \\
		
		光芒远射\ \ 万箭穿心\ \ 什么的关系\\
		导数两字\ \ 正负分析\ \ 何时有意义\\
		感情作伴\ \ 离合悲欢\ \ 谁人的默契\\
		黑白小技\ \ 金钱游戏\ \ 哪里的问题\\ \hspace*{\fill} \\
		
		光芒远射\ \ 万箭穿心\ \ 电荷的关系\\
		导数两字\ \ 正负分析\ \ 终究有意义\\
		感情作伴\ \ 离合悲欢\ \ 我们的默契\\
		黑白小技\ \ 金钱游戏\ \ 他们的问题\\ \hspace*{\fill} \\
		
		船帆起落\ \ 风鼓声\ \ 青春要靠岸\\
		莫念早先\ \ 莫想明天\ \ 人生无替换\\
		不必紧张\ \ 不必留恋\ \ 眼前的温暖\\
		水在心涧\ \ 笔在手中\ \ 梦想作舵盘\\ \hspace*{\fill} \\
		
		水在心涧\ \  笔在手中\ \  永远的舵盘
	\end{center}
	
	\hfill 2023年3月10日
	
	
	\subsection{十六字令四首}
	\begin{center}
		零!病去春来任我赢。媒空啭、死海浪波平。\\
		零!老理拙才苦著名。科学事、淡我笔诗情。\\
		零!宇内未尝半日宁。狐愚钝、徒守草株形。\\
		零!旧景新朝雪漫庭。犹难坠、烽火少年行。
		
	\end{center}
	\hfill 2023年3月18日
	
	\subsection{奥森游记}
	奥森之行,其谋久矣。自长者计事以来,许半年有奇。去年十月,余为作函,曰:“春至则景残,夏至则地热,冬至则风寒。惟秋日至,可赏无限风光,时令之狡黠,盖至于此。”然事以疫废。时冬销暖至,疫往春来,乃复议之,终可。三月二十五日日中往,至日暮之时方归。是日余空囿草地,以为兴之不尽,遂于次日约王子权再往,得见其景。于是心满意足。
	
	时序初春,地初暖,天适晴,寒尾驻,暖端迎。日临红路,人绕绿坪。苇蒲汇,兰桃兴。凫鸭在水,野燕游汀。远塔近屋,传出古今活气;新花老树,助长遐迩春情。疏密合织,勃然新生之叶;青白交绘,婉转初落之英。水上观之,苇丛翠,洲草青。彩舟有迹,微风无形。稀芳陆倚,厚木天擎。水渺渺兮如画,岸遥遥兮渐宁。浅浅日光,铭刻春色浅浅;盈盈碧波,满映笑面盈盈。
	
	或有三家四族、新朋故友,往来游戏,结伴心灵。席地论事,陈布说形。于是蜜水甘甜,滋于晨露;小食清爽,美于荤腥。飞盘神鹰,如入无人之地;沙袋老虎,似舞八佾之庭。树木慈立,仗枝叶蔽起日晒;花草喜笑,论生机还是人赢。
	
	余所记者固美,然其美者如是焉耳矣。其地非有烂漫之景,亦无壮丽之所,其未寡者,惟路而已。是故人多趋走园中,以为康健之道。余本欲为之,然鞋困脚乏,卒未尝得。
	
	时二〇二三年三月二十八日。
	\begin{flushright}
		2023年3月28日
	\end{flushright}
	
	\subsection[沁园春·定陵猫]{沁园春·定陵猫\footnote{白玉游定陵,见猫。又见游人遍掷钱币于陵内。有感而作。}}
	父祖烟波,四百年头,懈守定陵。爱碎光绵影,骚家礼客,新砖故瓦,昵草疏灵。朝舐吾绒,夕寻吾伴,一日七餐糕饭盈。天光到,仗墨尖棕掌,荡我幽情。
	
	茅堆故事零零,道妻子人间万事名。问王旌风雨,由来变换;堂碑犬马,如旧哭鸣。神鬼无形,死生有界,走肉枯荣大柱擎。金钱至,遂渺然君献,佑尔心晴。
	\begin{flushright}
		2023年4月11日
	\end{flushright}
	
	\subsection{永遇乐·用前韵赋十九岁}
	红日浮鹰,那堪长顾、天雨人际。野老由头,尧风舜月,在我参差意。禹门\footnote{辛弃疾《鹧鸪天·送廓之秋试》:“禹门已准桃花浪,月殿先收桂子香。”}蛛朽,灵台\footnote{指心。}腊冻,一口耳鼻腥气。酒应来,糟足醉腿,蹬开内卷生计。
	
	烟波定料,虚华年岁,久咒心帆难寂。渡海寻洲,淡移疏忘,何事因得绩。暖如谁问,会当学往,为汝烧炉营驿。云销处,长虹万里,与君共霁。
	\begin{flushright}
		2023年4月28日
	\end{flushright}
	
	\subsection{减字木兰花·十九岁儿童节作}
	\begin{center}
		人生票贵,草木蝴蝶无限味。\\
		老大风流,丑面他乡瘦业求。\\
		心高愿小,梦取寰间忧乐早。\\
		笑我无机,人不戚戚我自欺。
		
	\end{center}
	\hfill 2023年6月1日
	
	\subsection{咏偶}
	\begin{center}
		鼠目大如枣,兔牙齐在排。\\
		暖日安衾腹,寒宵入我怀。\\
		有命空无命,新孩渡老孩。\\
		明朝生死替,谁为乞君骸?
		
	\end{center}
	\hfill 2023年6月15日
	
	\subsection{解}
	\begin{center}
		野虎野狼因特网,阿猫阿狗讲台桌。\\
		少年易写关门志,老大难求隔夜波。\\
		人类新光前日死,马家\footnote{指马克思。}旧恨九州活。\\
		天堂苦闷眉无笑,地狱风流胃有魔。
		
	\end{center}
	\hfill 2023年6月20日
	
	\subsection{巧逢焦子作}
	\begin{center}
		风雨高三四季狂,月如幽火子如霜。\\
		浓烟雾里逢金兔,淡腑怀中透冷光。\\
		\textbf{鲤过龙门剥去锈,我为羊肉烤\footnote{谐音双关,“考”。}来香。}\\
		六月桥头总顾忆,英雄宝剑尚锋芒。
		
	\end{center}
	\begin{flushright}
		2023年6月27日
	\end{flushright}
	
	\subsection[鹧鸪天二首]{鹧鸪天二首\footnote{白玉阅游戏《候鸟》有感,作下二首。}}
	\subsubsection{呈叶}
	下笔寒窗银月乖,犹思前日百薇栽。\textbf{姊兄怯梦殊途痛,野弟孤行万里猜。}
	
	书纸贵,圣贤呆,蜗牛休艳隼鹰胎。今宵笔墨劳君侧,明日芳泽向雨开。
	
	\subsubsection{答柔}
	寂寞长空候鸟乖,乱花定欲芷兰栽\footnote{赵俸翊《以我》:“乱花渐欲芷兰栽”。}。\textbf{倦削情意支吾算,闲摆书籍燕尔猜。}
	
	宵夜淡,午晨呆,高三磨尽子离胎。会当长念荼毒咒,自信人生誓汝开。
	\begin{flushright}
		2023年7月15日
	\end{flushright}
	\begin{flushleft}
		\textbf{附白玉自写的上二首词的解析如下:}
	\end{flushleft}
	\begin{small}
		
		第一首是以女主的视角写给男主的;第二首是以男主的视角完成的和词,所以两首词的韵脚是一致的。
		
		\textbf{接下来的介绍可能涉及到《候鸟》的剧透,对游戏感兴趣的请酌情阅读。}
		
		首先,《候鸟》的结局是这样的:女主梁芷柔前往了樱华大学(原型为浙江大学)读书,而男主叶雨潇为了追赶她的脚步、也为了实现自己的梦想选择复读。那么这两首词就是写在女主在大学上学、男主在高四复读的这一阶段。
		
		描述了“写作背景”之后,我需要说明的是,当我在写这两首词的时候,有在尝试代入人物的视角去尽可能地分析人物具体情感。所以在下文中,读者应该可以看到这方面的一些迹象。
		
		我们先来看代入女主视角写给男主的这一篇(呈叶)。
		
		“下笔寒窗银月乖”,无疑是在写一个夜景,交代写作时间——这是老生常谈。但是为什么她会在“银月”下下笔呢?这就需要结合后文解释了,我们暂且按下不表;“犹思前日百薇栽”是游戏的剧情,即男主在女主的帮助下取得较大进步,从只能上二本学校的成绩逆袭到可以就读于本省唯一的985“百薇大学(原型为兰州大学)”的成绩。这些相对于后面的东西都是好理解的。
		
		“姊兄怯梦殊途痛,野弟孤行万里猜”,原作剧情中,女主曾自认为是男主的“姐姐”,而男主也接受了“弟弟”的称呼。“怯梦殊途痛”很好理解,就是梦见两个人走不到一起,然后害怕地从梦里惊醒,这当然也就是之前说的,女主看见了“银月乖”的原因;“孤行万里猜”也很好理解,就是一个人在学海中踽踽独行,并且心里还挂念着万里之外的另一个人。
		
		乍看之下这一联很简单,但是事实并非如此。试想,女主,也就是“姊兄”,难道不也是在“孤行万里猜”吗?而男主,也就是“野弟”,难道不会“怯梦殊途痛”吗?所以整体而言,这是一个\textbf{互文}的写法,就像《木兰诗》里的“将军百战死,壮士十年归”一样。结合视角,我们可以看到的是,女主寻找了两样最能和男主共情的因素,用细腻的笔法写出了一幅两个人彼此挂念的画面。
		
		另外,这一联还有虚实结合的成分。女主见不到男主,她只能通过想象,猜想她的“野弟”正在经历什么。细分的话,这是一个\textbf{对写}的写法。简而言之就是,我在想你,但是我不写我想你,而是写你想我。当然,对写是属于虚写的范畴的。但是由于这一联中,作者把两个人的“虚”和“实”利用互文混在了一起,所以虚虚实实,虚实难辨,感情真挚。
		
		“书纸贵,圣贤呆,蜗牛休艳隼鹰胎”,这一联就又涉及到游戏的内容。在游戏中,女主去过樱华(原型为浙江杭州)的大城市,曾向男主描述大城市的景象,感慨于家乡与大城市教育资源与学生眼界的区别,对自己的未来充满迷茫。在这里我融入了自己的理解。即,我认为当女主实地来到了樱华并就读于樱华的时候,她绝对不会比大多数的本地学生差。这样一来,这位眼界更宽的女主就可以去劝诫男主——“圣贤呆”,也就是说女主看到她自己曾经以为的、眼界开阔的“圣贤”们也并没有多么厉害。“蜗牛休艳隼鹰胎”则化用了一个故事,即“能达到山顶的两种动物,一种是出生就有飞行基因的‘隼鹰’;另一种是靠自己不懈努力的‘蜗牛’”,有劝诫男主的意味。至于“书纸贵”,则是对于大城市现象的一个描写。
		
		“今宵笔墨劳君侧,明日芳泽向雨开。”则相对简单。“雨”,指代男主“叶雨潇”。注意这个“向”字,之所以没有用“为”字,是因为我不想让努力和成功紧密地挂钩,而是想突出一个“芳泽”在等待“雨”的效果,在逻辑上也比较符合。
		
		然后我们来看男主给女主的和词(答柔)。
		
		“寂寞长空候鸟乖,乱花定欲芷兰栽”,前一句是写时令,大概是秋冬之际,候鸟已然无影无踪。“乱花定欲芷兰栽”,“芷兰”,毫无疑问是指代女主“梁芷柔”,而“乱花”指的又是什么呢?在我的理解中,如此优秀的男主可能会在复读过程中收获许多女生的芳心。所以这里可以理解为男主在向女主表决心,不去管那些“乱花”,而是一定要“芷兰”。
		
		“倦削情意支吾算,闲摆书籍燕尔猜”。我觉得这一联写得很好,同时也写得很烂。烂就在于它读起来不是很顺口,毕竟我现在的创作都是以“在符合格律的前提下尽可能读得顺口”为基本目标的。好则是在于它实际上是双关语,不过不是谐音的双关,而是纯粹语义上的双关。双关的接口就在于读者对“支吾”和“燕尔”的理解。
		
		在解释双关语之前,我们先理解一下“候鸟”这个题目的含义。根据女主在剧情中的说法,候鸟有“夏候鸟”和“冬候鸟”之分。每当夏候鸟到达一个地方的时候,冬候鸟已经出发很久了,所以说夏候鸟永远赶不上冬候鸟。女主害怕自己由于眼界的关系,在人生中只能做一只永远被甩在后面的“夏候鸟”,故出此语。
		
		回到词上来,从前一句开始解释,如果我们把“支吾”理解成支撑我,那么就是一个动宾的搭配。整句话的意思就可以理解成:当我困倦的时候,我把我们之间的情意削成“锥”,从而“刺股”,支撑着我继续算下去。那么这样一来,根据对偶的性质,我们也必须把后面的“燕尔”作为动宾短语解释。燕子,正好是一种夏候鸟。因此后面这句话的意思就是:当我清闲的时候,我摆弄着书本,想象着你正在像燕子这样的夏候鸟一样,奋力地追着冬候鸟的脚步。
		
		然而,如果我们把“支吾”和“燕尔”作为两个纯粹的形容词来理解的话,意思就不一样了。“支吾算”,就是写一步卡一步的算。所以前一句话意思就变成了:我们的情意在我困倦的时候支撑着我,就像“锥刺股”一样让我强迫自己继续写那些让我写起来很难受的题目。这一句看起来意思和原来区别不大,而后一句就完全不一样了。“燕尔”,作为汉语词语,意思是“夫妇之间和谐的样子”,所以有一个成语叫“新婚燕尔”。所以后一句话意思就变成了:在我清闲的时候,我摆弄着书本,想象着我们以后结为夫妇之时“燕尔”的样子。
		
		对比两种解读,或许你会觉得第二种更符合逻辑。但是在创作的过程中,我确实有想到“双关”这一点。毫不隐藏地说,当我写下“支吾”的时候,我是根据“吾”去寻找一个带“尔”的词,从而找到了“燕尔”,从而构成了这句话的双关。
		
		“宵夜淡,午晨呆,高三磨尽子离胎”,这一联就很好理解了。“宵夜”,可不是吃的那个“宵夜”,“宵”和“夜”都是夜晚的意思,所以“宵夜淡”指的是夜晚即将结束,意指男主熬夜学习,因此才会“午晨呆”;“高三磨尽子离胎”是写这次复读也是男主的最后一年高中生活,随后“子”就要“离胎”。因此,他有了最后一联的宣誓。
		
		“会当长念荼毒咒,自信人生誓汝开”,“荼毒咒”,出自在男主站台送女主时女主的话,这里就不解释具体内容了,感兴趣的读者可以去品鉴游戏。“自信人生誓汝开”就是宣誓了。事实上,我在写这句话的时候,一直没有敲定“誓”字。一开始我是想写一种“一起”的含义,所以用了“并”、“共”、“伴”、“渡”等,却都觉得不好,因为这些字会弱化男主誓言的强烈性;后又改为“携”,但是觉得读起来不顺口;又写了个“迈”,却觉得语义不通;最后敲定了还是用“誓言”的“誓”。
		
	\end{small}
	
	\subsection[只得一生]{只得一生\footnote{有借鉴罗大佑《只得一生》,也是为《候鸟》而写的。}}
	\begin{center}
		想问谁能冷却风雨兼程的握手\\
		想问谁能解救纷扰乌云的忧愁\\
		想问谁能改编命运交错的节奏\\
		让那雨声化成音符跳出生命的节奏\\ \hspace*{\fill} \\
		
		可又来到与谁辛苦告别的路口\\
		只怕挥挥手\  重逢却已白了头\\
		可否能够学习笑脸的潇洒通透\\
		忘记那清醒的梦诗歌的醋永恒的咒\\ \hspace*{\fill} \\
		
		走啊走\  让日月星辰伴我走\  该如何回头\\
		留啊留\  像山花草木一样留\  该如何解忧\\ \hspace*{\fill} \\
		
		想问谁能写出梦想漂泊的结构\\
		想问谁能晓得嫩芷兰花的温柔\\
		想问谁能挽住背影颤抖的双手\\
		让那冰雪化成暖雨融在冰冷的双手
	\end{center}
	
	
	
	\subsection{骑行长安街作}
	\begin{center}
		秋风多日劲,吹我到街南。\\
		赭路人光暖,黄车木影寒。\\
		老魂迁守易,新锐改赓难。\\
		遥想齐民袖,应拂梦里栏。
	\end{center}
	\begin{flushright}
		2023年9月26日
	\end{flushright}
	
	\subsection[明天会更好]{明天会更好\footnote{有借鉴罗大佑《明天会更好》。}}
	\begin{center}
		给我一支浑浊的蜡笔\\
		让我画出你的喘息\\
		那些黑白的烙印萧瑟的风景悲哀地埋在这里\\
		歌声扮演和平\   微笑扮演泪滴\\
		那人海涨潮的记忆\\
		与世界隔断了\\
		\hspace*{\fill} \\
		酒精驱散昨日的梦魇\\
		远方传来新的谎言\\
		声称前路的冰霜化作了清泉灌溉了一片花园\\
		双亲拉扯容颜\   春光如此的咸\\
		那得偿所愿的笑脸\\
		歪曲曾经的诺言\\
		\hspace*{\fill} \\
		良心将被侵蚀\   世界从未羞耻\\
		清醒的人最无知\\
		等谁赋予人生的价值\\
		风筝般的歌谣\   充当了灵魂的膏药\\
		谁仍然相信明天会更好\\
		\hspace*{\fill} \\
		谁愿岁月充满了教训\\
		谁愿人生过得小心\\
		谁能放手拨开千里的乌云寂寞地迎接天晴\\
		身躯占有心灵\   鸿毛一样地轻\\
		让金钱冷却了热情\\
		出租尊严与性命\\ 
		\hspace*{\fill} \\
		良心将被侵蚀\   世界从未羞耻\\
		清醒的人最无知\\
		等谁赋予人生的价值\\
		风筝般的歌谣\   充当了灵魂的膏药\\
		谁仍然相信明天会更好\\
		\hspace*{\fill} \\
		给我一支浑浊的蜡笔\\
		让我画出你的喘息\\
		那些凄凉的声音孤单的背影依然还埋在这里\\
		关心出于关系\   坚信如此艰辛\\
		让恩怨裹挟了咒语\\
		唤醒遥远的伴侣\\
		\hspace*{\fill} \\
		良心将被侵蚀\   世界从未羞耻\\
		清醒的人最无知\\
		等谁赋予人生的价值\\
		风筝般的歌谣\   充当了灵魂的膏药\\
		谁仍然相信明天会更好\\
		\hspace*{\fill} \\
		谁要我相信明天会更好?\\
		谁陪我相信明天会更好?
	\end{center}
	\hfill 2023年10月4日
	
	\subsection[念奴娇(长溪瘦马)]{念奴娇}
	长溪瘦马,算方足泪眼,股肱萧瑟。万里金秋都是梦,曾有天涯狂客。雨断新桥,风吹早絮,总是青云恶。光阴易老,景曦疏影难落。
	
	名岁天赐如歌,小儿尝待,蜜茧宏图破。何日倚天挥巨翼,吞尽往来寒热?似见当年,野山荒月,一片红旗阔。土衣尘面,映看灯火颜色。
	\begin{flushright}
		2023年10月17日
	\end{flushright}
	
	\subsection[水龙吟(计科无尽霜雷)]{水龙吟\footnote{白玉曾以此词竞选班长。在此之前,白玉为“雪绒花使者”(北师大对心理委员的称呼),故曰“雪花之后,几番澄澈”。}}
	计科无尽霜雷,小贤迟步征途阔。无怀有志,芳言绣笔,永耕难辍。写尽青年,忧愁日月,此心恒热。近西风渐起,每观又是、黄叶地,光阴客。
	
	白玉号名长握,到如今、凛然晶色。我生青翠,算来应抚,往来萧瑟。异或生涯,雪花之后,几番澄澈。愿荆棘乱海,浪花千里,与君同破!
	\begin{flushright}
		2023年10月30日
	\end{flushright}
	
	\subsection[木兰花(灰猫寂寞西天暮)]{木兰花}
	灰猫寂寞西天暮,颓卧懒登黄叶树。行无远伴但咸风,食有桂花陪苦醋。
	
	伤心萧瑟人间路,梦里芳华留不住。八方四舍虎狼豺,我自巢空回鸟处。
	\begin{flushright}
		2023年11月22日
	\end{flushright}
	
	\subsection{计算机网络结课赋}
	\begin{center}
		计网何寥廓,风流四海空。\\
		抓包深器外,阅码浅文中。\\
		老墨描云淡,新岚化雪浓。\\
		素衣今问事,纸上可盘龙。
	\end{center}
	\hfill 2023年12月10日
	
	\newpage
	\section{2024}
	
	\subsection{念奴娇·操作系统结课赋}
	计科系统,引机器会了、人间凉热。虚拟内存君莫问,都是进程轮廓。那里IO,谁家段页,裹碎平生客。预防死锁,老迪无功无过。
	
	数尽王道\footnote{计算机专业的考研书籍,最知名的品牌就叫“王道考研”。}征途,此肝此胆,潇洒宏图阔。安得两胁生素翼,奉我天鹅颜色?解透儒冠,当抛敝屣,学尔一航魄。小躯宏愿,自有灯火难落。
	\begin{flushright}
		2024年1月7日
	\end{flushright}
	
	\subsection[贺新郎(偶坐苍生椅)]{贺新郎}
	偶坐苍生椅。百余年、宏图只被,小鹰呼起。死去从来波滤尔,留有龙辞凤理。尘土后、无悲无喜。自古财钱达日月,便当时揉碎贫农米。天地梦,谁相抵?
	
	金山银水无穷已。但平生、朝阳易落,苦肠难洗。书尽英雄摧神木,淡了人间墨笔。空社稷、山河字里。铜齿基因今犹在,净蝰蛇咬断三千匹。平岁月,振梁脊。
	\begin{flushright}
		2024年2月1日
	\end{flushright}
	
	\subsection{西江月·用赵俸翊韵抒怀}
	料道生年常恨,应知死去如苏。人间福地远山崒,地底灵堂皆宿。
	
	少小笙歌游醉,老来畏望归途。青天顿首凤凰屋,问我梦回何处。
	\begin{flushright}
		2024年2月24日
	\end{flushright}
	
	\begin{flushleft}
		\textbf{附赵俸翊原词《西江月·自家返校赴学》:}
	\end{flushleft}
	
	 足算春歇良短,举踱冬絮如苏。梦隔南北越铁崒,万里山河晨宿。
	 
	两头寰楼碧宇,一条飞架天途。身迁远迩尽邻屋,谁问不知归处。
	
	\subsection[水调歌头·与初中同学聚餐赋]{水调歌头·与初中同学聚餐赋\footnote{聚餐时间为2024年2月8日。当日,白玉成句“背负青天月,歧路采花行”,后搁置。及3月3日,思之,遂毕之。}}
	背负青天日\footnote{《庄子·逍遥游》:“鹏之背,不知其几千里也,怒而飞,其翼若垂天之云。是鸟也,海运则将徙于南冥......背负青天,而莫之夭阏者,而后乃今将图南。”},歧路岭南行\footnote{黄庭坚《出迎使客质明放船自瓦窋归》:“风行水上如云过,地近岭南无雁来。”}。梦中霜雪弥尽,惟有物和情。洒遍青州从事\footnote{《世说新语·术解》:“桓公有主簿善别酒,有酒则令先尝,好者谓‘青州从事’,恶者谓‘平原督邮’。”},捻碎对床风雨\footnote{韦应物《示全真元常》:“宁知风雨夜,复此对床眠。”},早入尺规龄。长恨少年晚,愁至已浮萍。
	
	盟鸥志,当远避,堕心灵\footnote{《列子·黄帝》:“海上之人有好沤(通“鸥”)鸟者,每旦之海上从沤鸟游,沤鸟之至者,百住而不止。其父曰:‘吾闻沤鸟皆从汝游,汝取来吾玩之。’明日之海上,沤鸟舞而不下也。”此句之意并不是“当远避盟鸥志,(因为其)堕心灵”,而是“(我有)盟鸥志,(我)当远避堕心灵(之事)”。}。宏图如月,时运天地署功名。魂走刀旄剑羽,身卧龙席锦被,老去任君评。富贵来生事,聊寿海波宁\footnote{戚继光《韬钤深处》:“封侯非我意,但愿海波平。”}。
	\begin{flushright}
		2024年3月3日
	\end{flushright}
	
	\subsection[水调歌头(我菜不须练)]{水调歌头}
	我菜不须练,早悟此生空。天堂于我何有?离去自匆匆。咏叹古今咒语,听取人间况味,有句也无衷。忍遂青云便\footnote{柳永《鹤冲天》:“未遂风云便,争不恣狂荡”;曹雪芹《临江仙·柳絮》:“好风凭借力,送我上青云。”},开口骂仙翁。
	
	文章贱,忽回首,问东风:桃花流水随去,做了几焦功?待到寒窗苦尽,应似春花翠叶,摇曳火炉中。愿膑膝和脚,不跪孔家公\footnote{康有为曾在《请饬全国祀孔仍行跪拜礼》中叫嚣:“中国民不拜天,又不拜孔子,留此膝何为?”}。
	
	\begin{flushright}
		2024年3月18日
	\end{flushright}
	
	\subsection[念奴娇·曦园猫]{念奴娇·曦园\footnote{北京师范大学电子楼前有一片小花园,唤作“曦园”。}猫}
	铜楼外望,步春情小路,油光润雨。怀苦承穷多梦者,总把玫瑰奉予。通体铅华,善能奔跃,更有鸳鸯侣。笑痴风月,昧然余岁羁旅。
	
	来世我若为猫,野心无愧,肝胆清如许:愿弃此身空九命,就近都分人与。老大无仇,童孺无泪,不教寒鹰举。阳春三界,万家忧乐同语。
	
	\begin{flushright}
		2024年3月30日
	\end{flushright}
	
	
	
	
	
\end{document}
