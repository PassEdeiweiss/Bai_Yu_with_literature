\documentclass[a5paper]{ctexart}
\usepackage{graphicx, url, float}
\usepackage[fontset=none]{ctex}
\usepackage{geometry}
\usepackage{amssymb,amsmath}
\renewcommand{\d}{\mathop{}\!\mathrm{d}}
\newcommand{\e}{\mathrm{e}}
\renewcommand{\i}{\mathrm{i}}
\newcommand{\R}{\mathbb{R}}
\newcommand{\C}{\mathbb{C}}
\newcommand{\N}{\mathbb{N}}
\newcommand{\Z}{\mathbb{Z}}
\newcommand{\arsinh}{\operatorname{arsinh}}
\newcommand{\arcosh}{\operatorname{arcosh}}
\geometry{a5paper,left=2cm,right=2cm,top=2.5cm,bottom=2.5cm} 
\setcounter{secnumdepth}{4}
\setcounter{tocdepth}{4}


\title{\textbf{苏辛词导读}}
\author{}
\date{February 2024}


\begin{document}
	\maketitle
	\newpage
	\tableofcontents
	
	\newpage
	\section{苏轼词选}
	
	\subsection{行香子·过七里濑}
	一叶舟轻。双桨鸿惊。水天清、影湛波平。鱼翻藻鉴,鹭点烟汀。过沙溪急,霜溪冷,月溪明。
	
	重重似画,曲曲如屏。算当年、虚老严陵。君臣一梦,今古虚名。但远山长,云山乱,晓山青。
	
	\subsection{江城子}
	\begin{small}
		湖上与张先同赋,时闻弹筝。
	\end{small}
	
	凤凰山下雨初晴。水风清。晚霞明。一朵芙蕖,开过尚盈盈。何处飞来双白鹭,如有意,慕娉婷。
	
	忽闻江上弄哀筝。苦含情。遣谁听。烟敛云收,依约是湘灵。欲待曲终寻问取,人不见,数峰青。
	
	\subsection{瑞鷓鴣·觀潮}
	碧山影裡小紅旗。儂是江南蹋浪兒。拍手欲嘲山簡醉,齊聲爭唱浪婆詞。
	
	西興渡口帆初落,漁浦山頭日未敧。儂欲送潮歌底曲,尊前還唱使君詩。
	
	\subsection{昭君怨·金山送柳子玉}
	誰作桓伊三弄。驚破綠窗幽夢。新月與愁煙。滿江天。
	
	欲去又還不去。明日落花飛絮。飛絮送行舟。水東流。
	
	\subsection{减字木兰花}
	\begin{center}
		双龙对起。白甲苍髯烟雨里。\\
		疏影微香。下有幽人昼梦长。\\
		湖风清软。双鹊飞来争噪晚。\\
		翠飐红轻。时上凌霄百尺英。
	\end{center}
	
	
	\subsection{虞美人·有美堂贈述古}
	湖山信是東南美。一望彌千里。使君能得幾回來。便使尊前醉倒、更徘徊。
	
	沙河塘裡燈初上。水調誰家唱。夜闌風靜欲歸時。惟有一江明月、碧琉璃。
	
	\subsection{劝金船·流杯亭和杨元素}
	无情流水多情客。劝我如曾识。杯行到手休辞却。这公道难得。曲水池上,小字更书年月。还对茂林修竹,似永和节。 
	
	纤纤素手如霜雪。笑把秋花插。尊前莫怪歌声咽。又还是轻别。此去翱翔,遍赏玉堂金阙。欲问再来何岁,应有华发。 
	
	\subsection{定风波·送元素}
	千古风流阮步兵,平生游宦爱东平。千里远来还不住,归去,空留风韵照人清。
	
	红粉尊前添懊恼,休道,如何留得许多情,记取明年花絮乱,看泛,西湖总是断肠声。
	
	\subsection{永遇乐•寄孙巨源}
	\begin{small}
		一本有自序:孙巨源以八月十五日离海州。坐别于景疏楼上。既而与余会于润州。至楚州乃别。余以十一月十五日至海州。与太守会于景疏楼上。作此词以寄巨源。 
	\end{small}
	
	长忆别时,景疏楼下,明月如水。美酒清歌,留连不住,月随人千里。别来三度,孤光又满,冷落共谁同醉。卷珠帘,凄然顾影,共伊到明无寐。
	
	今朝有客,来从淮上,能道使君深意。凭仗清淮,分明到海,中有相思泪。而今何在,西垣清禁,夜永露华侵被。此时看,回廊晓月,也应暗记。
	
	\subsection{沁园春·赴密州早行马上寄子由}
	孤馆灯青,野店鸡号,旅枕梦残。渐月华收练,晨霜耿耿,云山摛锦,朝露漙漙。世路无穷,劳生有限,似此区区长鲜欢。微吟罢,凭征鞍无语,往事千端。
	
	当时共客长安。似二陆初来俱少年。有笔头千字,胸中万卷,致君尧舜,此事何难。用舍由时,行藏在我,袖手何妨闲处看。身长健,但优游卒岁,且斗尊前。
	
	\subsection{南乡子·梅花词和杨元素}
	寒雀满疏篱。争抱寒柯看玉蕤。忽见客来花下坐,惊飞。蹋散芳英落酒卮。
	
	痛饮又能诗。坐客无毡醉不知。花谢酒阑春到也,离离。一点微酸已着枝。
	
	\subsection{江城子·乙卯正月二十日夜记梦}
	十年生死两茫茫。不思量。自难忘。千里孤坟,无处话凄凉。纵使相逢应不识,尘满面,鬓如霜。 
	
	夜来幽梦忽还乡。小轩窗。正梳妆。相顾无言,惟有泪千行。料得年年断肠处,明月夜,短松冈。 
	
	\subsection{雨中花慢}
	今歲花時深院,盡日東風,輕颺茶煙。但有綠苔芳草,柳絮榆錢。聞道城西,長廊古寺,甲第名園。有國豔帶酒,天香染袂,為我留連。
	
	清明過了,殘紅無處,對此淚灑尊前。秋向晚,一枝何事,向我依然。高會聊追短景,清商不暇餘妍。不如留取,十分春態,付與明年。
	
	\subsection{江城子·猎词}
	老夫聊发少年狂。左牵黄。右擎苍。锦帽貂裘,千骑卷平冈。为报倾城随太守,亲射虎,看孙郎。
	
	酒酣胸胆尚开张。鬓微霜。又何妨。持节云中,何日遣冯唐。会挽雕弓如满月,西北望,射天狼。 
	
	\subsection{望江南·超然台作}
	春未老,风细柳斜斜。试上超然台上看,半壕春水一城花。烟雨暗千家。
	
	寒食后,酒醒却咨嗟。休对故人思故国,且将新火试新茶。诗酒趁年华。
	
	\subsection{水调歌头}
	\begin{small}
		丙辰中秋,欢饮达旦,大醉。作此篇,兼怀子由。
	\end{small}
	
	明月几时有,把酒问青天。不知天上宫阙,今夕是何年。我欲乘风归去,又恐琼楼玉宇,高处不胜寒。起舞弄清影,何似在人间。
	
	转朱阁,低绮户,照无眠。不应有恨,何事长向别时圆。人有悲欢离合,月有阴晴圆缺,此事古难全。但愿人长久,千里共婵娟。 
	
	\subsection{水调歌头}
	\begin{small}
		余去岁在东武,作水调歌头以寄子由。今年子由相从彭门百余日,过中秋而去,作此曲以别。余以其语过悲,乃为和之。其意以不早退为戒,以退而相从之乐为慰云。
	\end{small}
	
	安石在东海,从事鬓惊秋。中年亲友难别,丝竹缓离愁。一旦功成名遂,准拟东还海道,扶病入西州。雅志困轩冕,遗恨寄沧洲。
	
	岁云暮,须早计,要褐裘。故乡归去千里,佳处辄迟留。我醉歌时君和,醉倒须君扶我,惟酒可忘忧。一任刘玄德,相对卧高楼。
	
	\subsection{浣溪沙}
	\begin{small}
		徐门(一本作州)石谭谢雨,道上作五首。
	\end{small}
	
	照日深红暖见鱼。连溪绿暗晚藏乌。黄童白叟聚睢盱。麋鹿逢人虽未惯,猿猱闻喜不须呼。归家说与采桑姑。
	
	旋抹红妆看使君。三三五五棘篱门。相挨踏破茜罗裙。老幼扶携收麦社。乌鸢翔舞赛神村。道逢醉叟卧黄昏。
	
	麻叶层层檾叶光。谁家煮茧一村香。隔篱娇语络丝娘。垂白杖藜抬醉眼,捋青擣麨軟饥肠。问言豆叶几时黄。
	
	簌簌衣巾莎枣花。村南村北响缲车。牛衣古柳卖黄瓜。酒困路长惟欲睡,日高人渴漫思茶。敲门试问野人家。 
	
	软草平落过雨新。轻沙走马路无尘。何时收拾耦耕身。日暖桑麻光似泼,风来蒿艾气如薰。使君元是此中人。 
	
	\subsection{永遇樂}
	\begin{small}
		彭城夜宿燕子楼,梦盼盼,因作此词。
	\end{small}
	
	明月如霜,好風如水,清景無限。曲港跳魚,圓荷瀉露,寂寞無人見。紞如三鼓,鏗然一葉,黯黯夢雲驚斷。夜茫茫,重尋無處,覺來小園行遍。
	
	天涯倦客,山中歸路,望斷故園心眼。燕子樓空,佳人何在,空鎖樓中燕。古今如夢,何曾夢覺,但有舊歡新怨。異時對、黃樓夜景,為余浩歎。
	
	\subsection{卜算子·黄州定慧院寓居作}
	缺月挂疏桐,漏断人初静。时见幽人独往来,缥缈孤鸿影。 
	
	惊起却回头,有恨无人省。拣尽寒枝不肯栖,寂寞沙洲冷。 
	
	\subsection{水龙吟·次韵章质夫杨花词}
	似花还似非花,也无人惜从教坠。抛家傍路,思量却是,无情有思。萦损柔肠,困酣娇眼,欲开还闭。梦随风万里,寻郎去处,又还被、莺呼起。
	
	不恨此花飞尽,恨西园、落红难缀。晓来雨过,遗踪何在,一池萍碎。春色三分,二分尘土,一分流水。细看来,不是杨花点点,是离人泪。
	
	\subsection{水调歌头}
	\begin{small}
		欧阳文忠公尝问余:琴诗何者最善?答以退之听颖师琴诗。公曰:此诗固奇丽,然非听琴,乃听琵琶诗也。余深然之。建安章质夫家善琵琶者乞为歌词。余久不作,特取退之词,稍加檃括,使就声律以遗之云。
	\end{small}
	
	昵昵儿女语,灯火夜微明。恩冤尔汝来去,弹指泪和声。忽变轩昂勇士,一鼓填然作气,千里不留行。回首暮云远,飞絮搅青冥。
	
	众禽里,真彩凤,独不鸣。跻攀寸步千险,一落百寻轻。烦子指间风雨,置我肠中冰炭,起坐不能平。推手从归去,无泪与君倾。
	
	\subsection{南鄉子·重九涵輝樓呈徐君猷}
	霜降水痕收。淺碧鱗鱗露遠洲。酒力漸消風力軟,颼颼。破帽多情卻戀頭。
	
	佳節若為酬。但把清尊斷送秋。萬事到頭都是夢,休休。明日黃花蝶也愁。
	
	\subsection{水龍吟}
	\begin{small}
		閭秋大夫孝終公顯嘗守黃州,作棲霞樓,為郡中絕勝。元豐五年,余謫居黃。正月十七日夢扁舟渡江,中流回望,樓中歌樂雜作。舟中人言:公顯方會客也。覺而異之,乃作此曲。蓋越調鼓笛慢。公顯時已致仕。在蘇州.
	\end{small}
	
	小舟橫截春江,臥看翠壁紅樓起。雲間笑語,使君高會,佳人半醉。危柱哀絃,豔歌餘響,繞雲縈水。念故人老大,風流未減,空回首、煙波裡。
	
	推枕惘然不見,但空江、月明千里。五湖聞道,扁舟歸去,仍攜西子。雲夢南州,武昌東岸,昔遊應記。料多情夢裡,端來見我,也參差是。
	
	\subsection{江城子}
	\begin{small}
		陶渊明以正月五日游斜川,临流班坐,顾瞻南阜,爱曾城之独秀,乃作斜川诗,至今使人想见其处。元丰壬戍之春,余躬耕于东坡,筑雪堂居之。南挹四望亭之后丘,西控北山之微泉,慨然而叹,此亦斜川之游也。
	\end{small}
	
	梦中了了醉中醒。只渊明。是前生。走遍人间,依旧却躬耕。昨夜东坡春雨足,乌鹊喜,报新晴。 
	
	雪堂西畔暗泉鸣。北山倾。小溪横。南望亭丘,孤秀耸曾城。都是斜川当日境,吾老矣,寄余龄。 
	
	\subsection{定风波}
	\begin{small}
		三月七日,沙湖道中遇雨。雨具先去,同行皆狼狈,余独不觉。已而遂晴,故作此词。 
	\end{small}
	
	莫听穿林打叶声。何妨吟啸且徐行。竹杖芒鞋轻胜马。谁怕。一蓑烟雨任平生。 
	
	料峭春风吹酒醒。微冷。山头斜照却相迎。回首向来萧瑟处。归去。也无风雨也无晴。 
	
	\subsection{如梦令}
	为向东坡传语。人在玉堂深处。别后有谁来,雪压小桥无路。归去。归去。江上一犁春雨。
	
	\subsection{浣溪沙}
	\begin{small}
		遊蘄水清泉寺。寺臨蘭溪,溪水西流。
	\end{small}
	
	山下蘭芽短浸溪。松間沙路淨無泥。蕭蕭暮雨子規啼。
	
	誰道人生無再少?門前流水尚能西。休將白髮唱黃雞。
	
	\subsection{哨遍}
	\begin{small}
		陶渊明赋归去来,有其词而无其声。余既治东坡,筑雪堂于上,人俱笑其陋。读鄱阳董毅夫过而悦之,有卜邻之意。乃取归去来词,稍加檃括,使就声律,以遗毅夫。使家僮歌之。时相从于东坡,释耒而和之,扣牛角而为之节,不亦乐乎。
	\end{small}
	
	为米折腰,因酒弃家,口体交相累。归去来,谁不遣君归。觉从前、皆非今是。露未晞。征夫指余归路,门前笑语喧童稚。嗟旧菊都荒,新松暗老,吾年今已如此。但小窗、容膝闭柴扉。策杖看、孤云暮鸿飞。云出无心,鸟倦知还,本非有意。
	
	噫!归去来兮。我今忘我兼忘世。亲戚无浪语,琴书中、有真味。步翠麓崎岖,泛溪窈窕,涓涓暗谷流春水。观草木欣荣,幽人自感,吾生行且休矣。念寓形、宇内复几时。不自觉、皇皇欲何之。委吾心、去留谁计。神仙知在何处,富贵非吾志。但知临水登山啸咏,自引壶觞自醉。此生天命更何疑。且乘流、遇坎还止。
	
	\subsection{西江月}
	\begin{small}
		春夜行蕲山水中过酒家,饮酒醉,乘月至一溪桥上,解鞍曲肱醉卧少休。及觉已晓。乱山葱茏,流水锵然,疑非尘世也。书此语桥柱上。 
	\end{small}
	
	照野弥弥浅浪,横空隱隱層霄。障泥未解玉骢骄。我欲醉眠芳草。
	
	可惜一溪明月,莫教踏破琼瑶。解鞍欹枕绿杨桥。杜宇一声春晓。
	
	\subsection{念奴娇·赤壁怀古}
	大江东去,浪淘尽、千古风流人物。故垒西边,人道是、三国周郎赤壁。乱石穿空,惊涛拍岸,卷起千堆雪。江山如画,一时多少豪杰。
	
	遥想公瑾当年,小乔初嫁了,雄姿英发。羽扇纶巾,谈笑间、樯橹灰飞烟灭。故国神游,多情应笑我,早生华发。人生如梦,一尊还酹江月。
	
	\subsection{洞仙歌}
	\begin{small}
		仆七岁时,见眉山老尼,姓朱,忘其名,年九十余。自言:尝随其师入蜀主孟昶宫中。一日大热,蜀主与花蕊夫人夜起,避暑摩诃池上,作一词。朱具能记之。今四十年,朱已死,人无知此词者。独记其首两句,暇日寻味,岂《洞仙歌令》乎,乃为足之耳。
	\end{small}
	
	冰肌玉骨,自清凉无汗。水殿风来暗香满。绣帘开、一点明月窥人,人未寝、欹枕钗横鬓乱。
	
	起来携素手,庭户无声,时见疏星渡河汉。试问夜如何,夜已三更,金波淡、玉绳低转。但屈指、西风几时来,又不道、流年暗中偷换。
	
	\subsection{念奴嬌·中秋}
	憑高眺遠,見長空萬里,雲無留跡。桂魄飛來光射處,冷浸一天秋碧。玉宇瓊樓,乘鸞來去,人在清涼國。江山如畫,望中煙樹歷歷。
	
	我醉拍手狂歌,舉杯邀月,對影成三客。起舞徘徊風露下,今夕不知何夕。便欲乘風,翻然歸去,何用騎鵬翼。水晶宮裡,一聲吹斷橫笛。
	
	\subsection{醉翁操}
	\begin{small}
		琅琊幽谷,山川奇麗,泉鳴空澗,若中音會。醉翁喜之,把酒臨聽,輒欣然忘歸。既去十餘年,而好奇之士沈遵聞之。往遊,以琴寫其聲,曰醉翁操,節奏疏宕,而音指華暢,知琴者以為絕倫。然有其聲而無其辭。翁雖為作歌,而與琴聲不合。又依楚詞作醉翁引,好事者亦倚其辭以製曲。雖粗合韻度,而琴聲為詞所繩約,非天成也。後三十餘年,翁既捐館舍,遵亦沒久矣。有廬山玉澗道人崔閑,特妙於琴。恨此曲之無詞,乃譜其聲,而請東坡居士以補之云。
	\end{small}
	
	琅然。清圜。誰彈。響空山。無言。惟翁醉中知其天。月明風露娟娟。人未眠。荷蕢過山前。曰有心也哉此賢。
	
	醉翁嘯詠,聲和流泉。醉翁去後,空有朝吟夜怨。山有時而童巔。水有時而回川。思翁無歲年。翁今為飛仙。此意在人間。試聽徽外三兩絃。
	
	\subsection{滿庭芳}
	蝸角虛名,蠅頭微利,算來著甚乾忙。事皆前定,誰弱又誰強。且趁閒身未老,須放我、些子疏狂。百年裡,渾教是醉,三萬六千場。
	
	思量。能幾許,憂愁風雨,一半相妨。又何須,抵死說短論長。幸對清風皓月,苔茵展、雲幕高張。江南好,千鍾美酒,一曲滿庭芳。
	
	\subsection{临江仙·夜归临皋}
	夜饮东坡醒复醉,归来仿佛三更。家童鼻息已雷鸣。敲门都不应,倚杖听江声。
	
	长恨此身非我有,何时忘却营营。夜阑风静縠纹平。小舟从此逝,江海寄余生。
	
	\subsection{滿庭芳}
	\begin{small}
		有王長官者,棄官黃州三十三年,黃人謂之黃先生。因送陳慥來過余,因為賦此。
	\end{small}
	
	三十三年,今誰存者,算只君與長江。凜然蒼桧,霜幹苦難雙。聞道司州古縣,雲溪上、竹塢松窗。江南岸,不因送子,寧肯過吾邦。
	
	摐摐(chuāng)。疏雨過,風林舞破,煙蓋雲幢(chuáng)。願持此邀君,一飲空缸。居士先生老矣,真夢裡、相對殘釭(gāng)。歌舞斷,行人未起,船鼓已逄逄(páng)。
	
	\subsection{鹧鸪天}
	林断山明竹隐墙。乱蝉衰草小池塘。翻空白鸟时时见,照水红蕖细细香。
	
	村舍外,古城旁。杖藜徐步转斜阳。殷勤昨夜三更雨,又得浮生一日凉。
	
	\subsection{滿庭芳}
	\begin{small}
		元豐七年四月一日,余將去黃移汝,留別雪堂鄰里二三君子。會李仲覽自江東來別,遂書以遺之。
	\end{small}
	
	歸去來兮,吾歸何處,萬里家在岷峨。百年強半,來日苦無多。坐見黃州再閏,兒童盡、楚語吳歌。山中友,雞豚社酒,相勸老東坡。
	
	云何。當此去,人生底事,來往如梭。待閒看,秋風洛水清波。好在堂前細柳,應念我、莫翦柔柯。仍傳語,江南父老,時與曬漁蓑。
	
	\subsection{阮郎归·初夏}
	绿槐高柳咽新蝉。薰风初入弦。碧纱窗下水沉烟。棋声惊昼眠。
	
	微雨过,小荷翻。榴花开欲然。玉盆纤手弄清泉。琼珠碎却圆。
	
	\subsection{減字木蘭花}
	\begin{small}
		贈潤守許仲塗,且以「鄭容落籍、高瑩從良」為句首。
	\end{small}
	\begin{center}
		鄭莊好客。容我尊前先墮幘。\\
		落筆生風。籍籍聲名不負公。\\
		高山白早。瑩骨冰膚那解老。\\
		從此南徐。良夜清風月滿湖。
	\end{center}	
	
	\subsection{西江月·平山堂}
	三过平山堂下,半生弹指声中。十年不见老仙翁。壁上龙蛇飞动。
	
	欲吊文章太守,仍歌杨柳春风。休言万事转头空。未转头时皆梦。
	
	\subsection{如夢令}
	\begin{small}
		元豐七年十二月十八日浴泗州雍熙塔下,戲作如夢令兩闕。此曲本唐莊宗製,名憶仙姿,嫌其名不雅,故改為如夢令。莊宗作此詞,卒章云:「如夢。如夢。和淚出門相送。」因取以為名云。
	\end{small}
	
	水垢何曾相受。細看兩俱無有。寄語揩背人,盡日勞君揮肘。輕手。輕手。居士本來無垢。
	
	自淨方能淨彼。我自汗流呀氣。寄語澡浴人,且共肉身遊戲。但洗。但洗。俯為人間一切。
	
	\subsection{浣溪沙·元豐七年十二月二十四日從泗州劉倩叔遊南山}
	細雨斜風作小寒。淡煙疏柳媚晴灘。入淮清洛漸漫漫。
	
	雪沫乳花浮午琖,蓼茸蒿筍試春盤。人間有味是清歡。
	
	\subsection{定风波}
	\begin{small}
		王定国歌儿柔奴,姓宇文氏,眉目娟丽,善应对,家世住京师。定国南迁归,余问柔:“广南风土应是不好?”柔对曰:“此心安处便是吾乡。”因为缀词云。
	\end{small}
	
	谁羡人间琢玉郎。天应乞与点酥娘。尽道清歌传皓齿。风起。雪飞炎海变清凉。
	
	万里归来颜愈少。微笑。笑时犹带岭梅香。试问岭南应不好?却道。此心安处是吾乡。
	
	\subsection{西江月·寶雲真覺院賞瑞香}
	公子眼花亂發,老夫鼻觀先通。領巾飄下瑞香風。驚起謫仙春夢。
	
	后土祠中玉蕊,蓬萊殿後鞓紅。此花清絕更纖穠。把酒何人心動。
	
	\subsection{西江月·坐客見和復次韻}
	小院朱闌幾曲,重城畫鼓三通。更看微月轉光風。歸去香雲入夢。
	
	翠袖爭浮大白,皂羅半插斜紅。燈花零落酒花穠。妙語一時飛動。
	
	\subsection{西江月·再用前韻戲曹子方}
	\begin{small}
		傅本有“公自注云:坐客云瑞香為紫丁香。遂以此曲辨證之。”
	\end{small}
	
	怪此花枝怨泣,託君詩句名通。憑將草木記吳風。繼取相如雲夢。
	
	點筆袖沾醉墨,謗花面有慚紅。知君卻是為情穠。怕見此花撩動。
	
	\subsection{八声甘州·寄参寥子}
	有情风、万里卷潮来,无情送潮归。问钱塘江上,西兴浦口,几度斜晖。不用思量今古,俯仰昔人非。谁似东坡老,白首忘机。
	
	记取西湖西畔,正暮山好处,空翠烟霏。算诗人相得,如我与君稀。约他年、东还海道,愿谢公、雅志莫相违。西州路,不应回首,为我沾衣。
	
	\subsection{行香子}
	清夜無塵。月色如銀。酒斟時、須滿十分。浮名浮利,虛苦勞神。歎隙中駒,石中火,夢中身。
	
	雖抱文章,開口誰親。且陶陶、樂盡天真。幾時歸去,作箇閒人。對一張琴,一壺酒,一溪雲。
	
	\subsection{蝶恋花·春景}
	花褪残红青杏小。燕子飞时,绿水人家绕。枝上柳绵吹又少。天涯何处无芳草。
	
	墙里秋千墙外道。墙外行人,墙里佳人笑。笑渐不闻声渐悄。多情却被无情恼。
	
	\subsection{西江月·梅花}
	玉骨那愁瘴雾,冰姿自有仙风。海仙时遣探芳丛。倒挂绿毛么凤。
	
	素面翻嫌粉涴,洗妆不褪唇红。高情已逐晓云空。不与梨花同梦。
	
	\subsection{贺新郎·夏景}
	乳燕飞华屋。悄无人、桐阴转午,晚凉新浴。手弄生绡白团扇,扇手一时似玉。渐困倚、孤眠清熟。帘外谁来推绣户,枉教人、梦断瑶台曲。又却是,风敲竹。
	
	石榴半吐红巾蹙。待浮花、浪蕊都尽,伴君幽独。秾艳一枝细看取,芳心千重似束。又恐被、秋风惊绿。若待得君来向此,花前对酒不忍触。共粉泪,两簌簌。
	
	\subsection{西江月}
	世事一場大夢,人生幾度秋涼。夜來風葉已鳴廊。看取眉頭鬢上。
	
	酒賤常愁客少,月明多被雲妨。中秋誰與共孤光。把盏淒然北望。
	
	\subsection{千秋歲·次韻少遊}
	島邊天外。未老身先退。珠淚濺,丹衷碎。聲搖蒼玉佩。色重黃金帶。一萬里。斜陽正與長安對。
	
	道遠誰云會。罪大天能蓋。君命重,臣節在。新恩猶可覬。舊學終難改。吾已矣。乘桴且恁浮於海。
	
	\newpage
	\section{辛弃疾词选}
	\subsection{念奴娇·登建康赏心亭呈史致道留守}
	我来吊古,上危楼、赢得闲愁千斛。虎踞龙蟠何处是,只有兴亡满目。柳外斜阳,水边归鸟,陇上吹乔木。片帆西去,一声谁喷霜竹。
	
	却忆安石风流,东山岁晚,泪落哀筝曲。儿辈功名都付与,长日惟消棋局。宝镜难寻,碧云将暮,谁劝杯中绿。江头风怒,朝来波浪翻屋。
	
	\subsection{声声慢·滁州旅次登奠枕楼作,和李清宇韵}
	征埃成阵,行客相逢,都道幻出层楼。指点檐牙高处,浪拥云浮。今年太平万里,罢长淮、千骑临秋。凭栏望,有东南佳气,西北神州。
	
	千里怀嵩人去,应笑我、身在楚尾吴头。看取弓刀,陌上车马如流。从今赏心乐事,剩安排、酒令诗筹。华胥梦,愿年年、人似旧游。
	
	\subsection{水龙吟·登建康赏心亭}
	楚天千里清秋,水随天去秋无际。遥岑远目,献愁供恨,玉簪螺髻。落日楼头,断鸿声里,江南游子。把吴钩看了,栏干拍遍,无人会、登临意。
	
	休说鲈鱼堪鲙。尽西风、季鹰归未。求田问舍,怕应羞见,刘郎才气。可惜流年,忧愁风雨,树犹如此。倩何人,唤取盈盈翠袖,搵英雄泪。
	
	\subsection{菩萨蛮·书江西造口壁}
	郁孤台下清江水。中间多少行人泪。西北望长安。可怜无数山。
	
	青山遮不住。毕竟东流去。江晚正愁予。山深闻鹧鸪。
	
	\subsection{贺新郎}
	高阁临江渚。访层城、空馀旧迹,黯然怀古。画栋珠帘当日事,不见朝云暮雨。但遗意、西山南浦。天宇修眉浮新绿,映悠悠、潭影长如故。空有恨,奈何许。
	
	王郎健笔夸翘楚。到如今、落霞孤鹜,竞传佳句。物换星移知几度,梦想珠歌翠舞。为徙倚、阑干凝伫。目断平芜苍波晚,快江风、一瞬澄襟暑。谁共饮,有诗侣。
	
	\subsection{满江红·游南岩和范廓之韵}
	笑拍洪崖,问千丈、翠岩谁削?依旧是、西风白鸟,北村南郭。似整复斜僧屋乱,欲吞还吐林烟薄。觉人间、万事到秋来,都摇落。
	
	呼斗酒,同君酌。更小隐,寻幽约。且丁宁休负,北山猿鹤。有鹿从渠求鹿梦,非鱼定未知鱼乐。正仰看、飞鸟却应人,回头错。
	
	\subsection{水龙吟·过南剑双溪楼}
	举头西北浮云,倚天万里须长剑。人言此地,夜深长见,斗牛光焰。我觉山高,潭空水冷,月明星淡。待燃犀下看,凭栏却怕,风雷怒,鱼龙惨。
	
	峡束沧江对起,过危楼、欲飞还敛。元龙老矣,不妨高卧,冰壶凉簟。千古兴亡,百年悲笑,一时登览。问何人又卸,片帆沙岸,系斜阳缆?
	
	\subsection{汉宫春·会稽蓬莱阁观雨}
	秦望山头,看乱云急雨,倒立江湖。不知云者为雨,雨者云乎。长空万里,被西风、变灭须臾。回首听,月明天籁,人间万窍号呼。
	
	谁向若耶溪上,倩美人西去,麋鹿姑苏。至今故国人望,一舸归欤。岁云暮矣,问何不、鼓瑟吹竽。君不见,王亭谢馆,冷烟寒树啼乌。
	
	\subsection{汉宫春·会稽秋风亭怀古}
	亭上秋风,记去年袅袅,曾到吾庐。山河举目虽异,风景非殊。功成者去,觉团扇、便与人疏。吹不断,斜阳依旧,茫茫禹迹都无。
	
	千古茂陵词在,甚风流章句,解拟相如。只今木落江冷,眇眇愁余。故人书报,莫因循、忘却蓴鲈。谁念我,新凉灯火,一编太史公书。
	
	\subsection{南乡子·登京口北固亭有怀}
	何处望神州。满眼风光北固楼。千古兴亡多少事,悠悠。不尽长江衮衮流。
	
	年少万兜鍪。坐断东南战未休。天下英雄谁敌手。曹刘。生子当如孙仲谋。
	
	\subsection{永遇乐·京口北固亭怀古}
	千古江山,英雄无觅,孙仲谋处。舞榭歌台,风流总被,雨打风吹去。斜阳草树,寻常巷陌,人道寄奴曾住。想当年,金戈铁马,气吞万里如虎。
	
	元嘉草草,封狼居胥,赢得仓皇北顾。四十三年,望中犹记,烽火扬州路。可堪回首,佛狸祠下,一片神鸦社鼓。凭谁问,廉颇老矣,尚能饭否。
	
	\subsection{沁园春·带湖新居将成}
	三径初成,鹤怨猿惊,稼轩未来。甚云山自许,平生意气,衣冠人笑,抵死尘埃。意倦须还,身闲贵早,岂为莼羹鲈脍哉!秋江上,看惊弦雁避,骇浪船回。
	
	东冈更葺茅斋。好都把轩窗临水开。要小舟行钓,先应种柳,疏篱护竹,莫碍观梅。秋菊堪餐,春兰可佩,留待先生手自栽。沉吟久,怕君恩未许,此意徘徊。
	
	\subsection{踏莎行·赋稼轩,集经句}
	进退存亡,行藏用舍。小人请学樊须稼。衡门之下可栖迟,日之夕矣牛羊下。
	
	去卫灵公,遭桓司马。东西南北之人也。长沮桀溺耦而耕,丘何为是栖栖者。
	
	\subsection{水龙吟·题瓢泉}
	稼轩何必长贫,放泉檐外琼珠泻。乐天知命,古来谁会,行藏用舍。人不堪忧,一瓢自乐,贤哉回也。料当年曾问,饭蔬饮水,何为是、栖栖者。
	
	且对浮云山上,莫匆匆、去流山下。苍颜照影,故应流落,轻裘肥马。绕齿冰霜,满怀芳乳,先生饮罢。笑挂瓢风树,一鸣渠碎,问何如哑。
	
	\subsection{沁园春·再到期思卜筑}
	一水西来,千丈晴虹,十里翠屏。喜草堂经岁,重来社老,斜川好景,不负渊明。老鹤高飞,一枝投宿,长笑蜗牛戴屋行。平章了,待十分佳处,著个茅亭。
	
	青山意气峥嵘。似为我归来妩媚生。解频教花鸟,前歌後舞,更催云水,暮送朝迎。酒圣诗豪,可能无势,我乃而今驾驭卿。清溪上,被山灵却笑,白发归耕。
	
	\subsection{水调歌头·赋松菊堂}
	渊明最爱菊,三径也栽松。何人收拾,千载风味此山中。手把离骚读遍,自扫落英餐罢,杖屦晓霜浓。皎皎太独立,更插万芙蓉。
	
	水潺湲,云澒洞,石巃嵸。素琴浊酒唤客,端有古人风。却怪青山能巧,政尔横看成岭,转面已成峰。诗句得活法,日月有新工。
	
	\subsection{清平乐·检校山园书所见}
	连云松竹,万事从今足。拄杖东家分社肉,白酒床头初熟。
	
	西风梨枣山园,儿童偷把长竿。莫遣旁人惊去,老夫静处闲看。
	
	\subsection{满江红·山居即事}
	几个轻鸥,来点破、一泓澄绿。更何处、一双鸂鶒,故来争浴。细读离骚还痛饮,饱看修竹何妨肉。有飞泉、日日供明珠,三千斛。
	
	春雨满,秧新谷。闲日永,眠黄犊。看云连麦垄,雪堆蚕簇。若要足时今足矣,以为未足何时足。被野老、相扶入东园,枇杷熟。
	
	\subsection{鹧鸪天·黄沙道中}
	句里春风正剪裁。溪山一片画图开。轻鸥自趁虚船去,荒犬还迎野妇回。
	
	松菊竹,翠成堆。要擎残雪斗疏梅。乱鸦毕竟无才思,时把琼瑶蹴下来。
	
	\subsection{浣溪沙·泉湖道中赴闽宪,别瓢泉}
	细听春山杜宇啼。一声声是送行诗。朝来白鸟背人飞。
	
	对郑子真岩石卧,趁陶元亮菊花期。而今堪诵北山移。
	
	\subsection{行香子·山居客至}
	白露园蔬。碧水溪鱼。笑先生、网钓还锄。小窗高卧,风展残书。看北移山,盘谷序,辋川图。
	
	白饭青刍。赤脚长须。客来时、酒尽重沽。听风听雨,吾爱吾庐。笑本无心,刚自瘦,此君疏。
	
	\subsection{鹧鸪天}
	石壁虚云积渐高。溪声绕屋几周遭。自从一雨花零乱,却爱微风草动摇。
	
	呼玉友,荐溪毛。殷勤野老苦相邀。杖藜忽避行人去,认是翁来却过桥。
	
	\subsection{行香子·云岩道中}
	云岫如簪。野涨挼蓝。向春阑、绿醒红酣。青裙缟袂,两两三三。把曲生禅,玉版句,一时参。
	
	拄杖弯环。过眼嵌岩。岸轻乌、白发鬖鬖。他年来种,万桂千杉。听小绵蛮,新格磔,旧呢喃。
	
	
	\subsection{鹧鸪天·游鹅湖醉书酒家壁}
	春入平原荠菜花,新耕雨后落群鸦。多情白发春无奈,晚日青帘酒易赊。
	
	闲意态,细生涯,牛栏西畔有桑麻。青裙缟袂谁家女?去趁蚕生看外家。
	
	\subsection{鹧鸪天·戏题村舍}
	鸡鸭成群晚未收,桑麻长过屋山头。有何不可吾方羡,要底都无饱便休。
	
	新柳树,旧沙洲,去年溪打那边流。自言此地生儿女,不嫁金家即聘周。
	
	\subsection{清平乐·村居}
	茅檐低小,溪上青青草。醉里吴音相媚好,白发谁家翁媪。
	
	大儿锄豆溪东,中儿正织鸡笼,最喜小儿无赖,溪头卧剥莲蓬。
	
	\subsection{鹧鸪天·代人赋}
	陌上柔桑破嫩芽,东邻蚕种已生些。平冈细草鸣黄犊,斜日寒林点暮鸦。
	
	山远近,路横斜,青旗沽酒有人家。城中桃李愁风雨,春在溪头荠菜花。
	
	\subsection{清平乐·博山道中即事}
	柳边飞鞚。露湿征衣重。宿鹭惊窥沙影动。应有鱼虾入梦。
	
	一川淡月疏星。浣沙人影娉婷。笑背行人归去,门前稚子啼声。
	
	\subsection{西江月·夜行黄沙道中}
	明月别枝惊鹊,清风半夜鸣蝉。稻花香里说丰年,听取蛙声一片。
	
	七八个星天外,两三点雨山前。旧时茅店社林边,路转溪头忽见。
	
	\subsection{浣溪沙}
	父老争言雨水匀,眉头不似去年颦。殷勤谢却甑中尘。
	
	啼鸟有时能劝客,小桃无赖已撩人。梨花也作白头新。
	
	\subsection{浣溪沙·常山道中即事}
	北陇田高踏水频。西溪禾早已尝新。隔墙沽酒煮纤鳞。
	
	忽有微凉何处雨,更无留影霎时云。卖瓜声过竹边村。
	
	\subsection{沁园春·将止酒、戒酒杯使勿近}
	杯汝来前,老子今朝,点检形骸。甚长年抱渴,咽如焦釜,于今喜睡,气似奔雷。汝说“刘伶,古今达者,醉后何妨死便埋。”浑如此,叹汝於知已,真少恩哉。
	
	更凭歌舞为媒。算合作人间鸩毒猜。况怨无大小,生於所爱,物无美恶,过则为灾。与汝成言:“勿留亟退,吾力犹能肆汝杯。”杯再拜,道“麾之即去,招则须来”。
	
	\subsection{沁园春}
	\begin{small}
		城中诸公载酒入山,余不得以止酒为解,遂破戒一醉,再用韵.
	\end{small}
	
	杯汝知乎,酒泉罢侯,鸱夷乞骸。更高阳入谒,都称齑臼,杜康初筮,正得云雷。细数从前,不堪余恨,岁月都将曲糵埋。君诗好,似提壶却劝,沽酒何哉。
	
	君言病岂无媒。似壁上雕弓蛇暗猜。记醉眠陶令,终全至乐,独醒屈子,未免沉灾。欲听公言,惭非勇者,司马家儿解覆杯。还堪笑,借今宵一醉,为故人来。
	
	\subsection{西江月·遣兴}
	醉里且贪欢笑,要愁那得工夫。近来始觉古人书。信著全无是处。
	
	昨夜松边醉倒,问松我醉何如。只疑松动要来扶。以手推松曰去。
	
	\subsection{水调歌头}
	\begin{small}
		将迁新居不成,有感戏作。时以病止酒,且遣去歌者。末章及之。
	\end{small}

	我亦卜居者,岁晚望三闾。昂昂千里,泛泛不作水中凫。好在书携一束,莫问家徒四壁,往日置锥无。借车载家具,家具少於车。
	
	舞乌有,歌亡是,饮子虚。二三子者爱我,此外故人疏。幽事欲论谁共,白鹤飞来似可,忽去复何如。众鸟欣有托,吾亦爱吾庐。
	
	\subsection{添字浣溪沙·三山戏作}
	记得瓢泉快活时。长年耽酒更吟诗。蓦地捉将来断送,老头皮。
	
	绕屋人扶行不得,闲窗学得鹧鸪啼。却有杜鹃能劝道,不如归。
	
	\subsection{卜算子·齿落}
	刚者不坚牢,柔底难摧挫。不信张开口角看,舌在牙先堕。
	
	已阙两边厢,又豁中间个。说与儿曹莫笑翁,狗窦从君过。
	
	\subsection{夜游宫·苦俗客}
	几个相知可喜。才厮见、说山说水。颠倒烂熟只这是。怎奈向,一回说,一回美。
	
	有个尖新底。说底话、非名即利。说得口乾罪过你。且不罪,俺略起,去洗耳。
	
	\subsection{玉楼春·戏赋云山}
	何人半夜推山去?四面浮云猜是汝。常时相对两三峰,走遍溪头无觅处。
	
	西风瞥起云横渡,忽见东南天一柱。老僧拍手笑相夸,且喜青山依旧住。
	
	\subsection{西江月·以家事付儿曹示之}
	万事云烟忽过,百年蒲柳先衰。而今何事最相宜?宜醉宜游宜睡。
	
	早趁催科了纳,更量出入收支。乃翁依旧管些儿,管竹管山管水。
	
	\subsection{临江仙·侍者阿钱将行赋钱字以赠之}
	一自酒情诗兴懒,舞裙歌扇阑珊。好天良夜月团团。杜陵真好事,留得一钱看。
	
	岁晚人欺程不识,怎教阿堵留连。杨花榆荚雪漫天。从今花影下,只看绿苔圆。
	
	\subsection{永遇乐·戏赋辛字送茂嘉十二弟赴调}
	烈日秋霜,忠肝义胆,千载家谱。得姓何年,细参辛字,一笑君听取。艰辛做就,悲辛滋味,总是辛酸辛苦。更十分,向人辛辣,椒桂捣残堪吐。
	
	世间应有,芳甘浓美,不到吾家门户。比著儿曹,累累却有,金印光垂组。付君此事,从今直上,休忆对床风雨。但赢得,靴纹绉面,记余戏语。
	
	\subsection{摸鱼儿}
	\begin{small}
		淳熙己亥,自湖北漕移湖南,同官王正之置酒小山亭,为赋。
	\end{small}
	
	更能消、几番风雨。匆匆春又归去。惜春长恨花开早,何况落红无数。春且住。见说道、天涯芳草迷归路。怨春不语。算只有殷勤,画檐蛛网,尽日惹飞絮。
	
	长门事,准拟佳期又误。蛾眉曾有人妒。千金纵买相如赋,脉脉此情谁诉。君莫舞。君不见、玉环飞燕皆尘土。闲愁最苦。休去倚危楼,斜阳正在,烟柳断肠处。
	
	\subsection{祝英台近·晚春}
	宝钗分,桃叶渡。烟柳暗南浦。怕上层楼,十日九风雨。断肠片片飞红,都无人管,更谁劝、啼莺声住。
	
	鬓边觑。试把花卜归期,才簪又重数。罗帐灯昏,呜咽梦中语:是他春带愁来,春归何处,却不解带将愁去。
	
	\subsection{满江红}
	点火樱桃,照一架荼蘼如雪。春正好、见龙孙穿破,紫苔苍壁。乳燕引雏飞力弱,流莺唤友娇声怯。问春归、不肯带愁归,肠千结。
	
	层楼望,春山叠。家何在,烟波隔。把古今遗恨,向他谁说。蝴蝶不传千里梦,子规叫断三更月。听声声、枕上劝人归,归难得。
	
	\subsection{满江红·暮春}
	家住江南,又过了、清明寒食。花径里、一番风雨,一番狼藉。红粉暗随流水去,园林渐觉清阴密。算年年、落尽刺桐花,寒无力。
	
	庭院静,空相忆。无说处,闲愁极。怕流莺乳燕,得知消息。尺素如今何处也?彩云依旧无踪迹。谩教人、羞去上层楼,平芜碧。
	
	\subsection{满江红·暮春}
	可恨东君,把春去春来无迹。便过眼等闲输了,三分之一。昼永暖翻红杏雨,风晴扶起垂杨力。更天涯芳草最关情,烘残日。
	
	湘浦岸,南塘驿。恨不尽,愁如织。算年年孤负,对他寒食。便恁归来能几许,风流早已非畴昔。凭画栏一线数飞鸿,沉空碧。
	
	\subsection{满江红}
	紫陌飞尘,望十里、雕鞍绣毂。春未老、已惊台榭,瘦红肥绿。睡雨海棠犹倚醉,舞风杨柳难成曲。问流莺、能说故园无,曾相熟。
	
	岩泉上,飞凫浴。巢林下,栖禽宿。恨荼蘼开晚,谩翻船玉。莲社岂堪谈昨梦,兰亭何处寻遗墨。但羁怀、空自倚秋千,无心蹴。
	
	\subsection{念奴娇·书东流村壁}
	野棠花落,又匆匆、过了清明时节。刬地东风欺客梦,一枕云屏寒怯。曲岸持觞,垂杨系马,此地曾轻别。楼空人去,旧游飞燕能说。
	闻道绮陌东头,行人长见,帘底纤纤月。旧恨春江流不断,新恨云山千叠。料得明朝,尊前重见,镜里花难折。也应惊问:近来多少华发!
	
	\subsection{声声慢·嘲(一本作赋)红木犀}
	\begin{small}
		余儿时尝入京师禁中凝碧池,因书当时所见。
	\end{small}
	
	开元盛日,天上栽花,月殿桂影重重。十里芬芳,一枝金粟玲珑。管弦凝碧池上,记当时、风月愁侬。翠华远,但江南草木,烟锁深宫。
	
	只为天姿冷澹,被西风酝酿,彻骨香浓。枉学丹蕉,叶展偷染娇红。道人取次装束,是自家、香底家风。又怕是,为凄凉、长在醉中。
	
	\subsection{贺新郎·赋水仙}
	云卧衣裳冷。看萧然、风前月下,水边幽影。罗袜尘生凌波去,汤沐烟江万顷。爱一点、娇黄成晕。不记相逢曾解佩,甚多情、为我香成阵。待和泪,收残粉。
	
	灵均千古怀沙恨。恨当时、匆匆忘把,此仙题品。烟雨凄迷僝僽(chán zhòu)损,翠袂摇摇谁整。谩写入、瑶琴幽愤。弦断招魂无人赋,但金杯的皪银台润。愁殢酒,又独醒。
	
	\subsection{贺新郎·赋琵琶}
	凤尾龙香拨,自开元《霓裳》曲罢,几番风月?最苦浔阳江头客,画舸亭亭待发。记出塞、黄云堆雪。马上离愁三万里,望昭阳、宫殿孤鸿没,弦解语,恨难说。
	
	辽阳驿使音尘绝,琐窗寒、轻拢慢捻,泪珠盈睫。推手含情还却手,一抹《梁州》哀彻。千古事、云飞烟灭。贺老定场无消息,想沉香亭北繁华歇。弹到此,为呜咽。
	
	\subsection{踏莎行·赋木犀}
	弄影阑干,吹香岩谷。枝枝点点黄金粟。未堪收拾付薰炉,窗前且把离骚读。
	
	奴仆葵花,儿曹金菊。一秋风露清凉足。傍边只欠个姮娥,分明身在蟾宫宿。
	
	\subsection{虞美人·赋荼}
	群花泣尽朝来露。争奈春归去。不知庭下有荼䕷。偷得十分春色、怕春知。
	
	淡中有味清中贵。飞絮残英避。露华微渗玉肌香。恰似杨妃初试、出兰汤。
	
	\subsection{瑞鹤仙·赋梅}
	雁霜寒透幕。正护月云轻,嫩冰犹薄。溪奁照梳掠。想含香弄粉,艳妆难学。玉肌瘦弱。更重重、龙绡衬著。倚东风,一笑嫣然,转盼万花羞落。
	
	寂寞。家山何在,雪後园林,水边楼阁。瑶池旧约。鳞鸿更仗谁托。粉蝶儿只解,寻桃觅柳,开遍南枝未觉。但伤心,冷落黄昏,数声画角。
	
	\subsection{念奴娇·赋傅岩叟香月堂两花}
	未须草草,赋梅花,多少骚人词客。总被西湖林处士,不肯分留风月。疏影横斜,暗香浮动,把断春消息。试将花品,细参今古人物。
	
	看取香月堂前,岁寒相对,楚两龚之洁。自与诗家成一种,不系南昌仙籍。怕是当年,香山老子,姓白来江国。谪仙人,字太白还又名白。
	
	\subsection{最高楼·客有败棋者,代赋梅}
	花知否,花一似何郎。又似沈东阳:瘦棱棱地天然白,冷清清地许多香。笑东君,还又向,北枝忙。
	
	著一阵霎时间底雪。更一个缺些儿底月。山下路,水边墙。风流怕有人知处,影儿守定竹旁厢。且饶他,桃李趁,少年场。
	
	\subsection{最高楼·用韵答赵晋臣敷文}
	花好处,不趁绿衣郎。缟袂立斜阳。面皮儿上因谁白,骨头儿里几多香。尽饶他,心似铁,也须忙。
	
	甚唤得、雪来白倒雪。更唤得,月来香杀月。谁立马,更窥墙。将军止渴山南畔,相公调鼎殿东厢。忒高才,经济地,战争场。
	
	
	
\end{document}